\documentclass[green]{LRSguildcamp1}
\begin{document}
\name{\gPR{}}

Upholding the public image of the \cHeroLeague{\intro} is of utmost importance to superheroes! 
Dragging the \cHeroLeague{}'s name through the mud is of high interest to all villains. Either way, gossip is power. 

The reputation of the \cHeroLeague{\intro} is currently at a +5, which neither good nor bad. However, if it goes below 0 the \cHeroLeague{} is going have a PR crisis. 

Heroes or Villains who have declared for either side may call their HQ and ask their PR department to take a story to the press. This option is only available to characters who have declared for a team, but once you are declared there is no limitation on how many stories you can leak. There is also no limitation to what you can leak, but we ask that player exercise their common sense and keep it plausible. They must leak the truth or things that they have heard. 

The characters themselves have very minimal control over how the story breaks, but the way the story comes out will always benefit the Team (H/V) breaking the news. News will be broken in the order that the GMs receive stories. Even if the GMs receive the same story twice, both stories will be published, but with diminishing effect. 

To break a story you do the following:
\begin{enumerate}
\item Write down the story on a white sheet labeled Press Release that exists for this purpose. Do not be profligate! 
\item Write down your name as the person breaking the story. You may not lie. 
\item Roll a d6. Then add any modifiers you have and record this number as "Spin"
\item Deposit the press release in the envelope attached to the sign for your HQ. If you aren't part of a team, you can't break a story unless you have another way to get information out. 
\item Tell a GM you have sent off a press release.
The GM will then modify a "Public Opinion" number based on the spin of the story and publicly posts the story on a "news board".
\end{enumerate}

%Heroes want to bring that number up, Villains want to bring that number down.  On the "You are a hero" blue sheet you add a goal that says "keep Public Opinion of the Hero league high," and to the villains that they want to lower that number.  And possibly specify that whichever hero wants to break the real story about the coverup of the Chicago incident values that above Public opinion of the hero league.
%Modifiers to PR Mechanic:
%Values that can be rolled on the die range from 1-6. With modifiers, you could technically get values as low as -2 or as high as 8; we'd list a full chart with all of the options and their impact. We publicly list the impact of each die value (i.e.: a value of 2 on the die is worth -2 points to the PO number.)
%Submitting a story to your HQ automatically provides a modifier that benefits that group's goal (Hero league is +1 to die roll, Villain compact is -1 to die roll).
%Submitting a story through Tween or YS gives them the option to +/- 1 from the die roll at their discretion.

\end{document}
