\documentclass[green]{LRSguildcamp1}
\begin{document}
\name{\gInternetPR{}}

As \cTween{} the internet extraordinaire you are an expert when it comes to the Internet. However, you have to try and keep this on the down low otherwise the family might connect this to your IntheKnow activities. The less people that know, the better. 

Ability to ask the internet - Usable three times

For The Super Know website...you can use your contacts online to confirm, track down, and figure out information. Write down your question answered or the fact you want confirmed on a piece of paper. Roll a d6 six. Multiple the result by 5. Write this number also on the paper. Give the paper to a GM. Set a timer for this number of minutes. When your timer goes off, return the the GM for your answer. 

Publish on the Internet - Unlimited uses.

You can break stories through the Internet! At first they will be whatever stories you want to publish, even if others find out about your articles you still belive the Internet has a right to know and will continue to publish. 

To publish on the internet, anyone can approach you and get you to post something to the internet under your pseudonym, but you have control over the Spin of the story. 

To break a story via the Internet you do the following:
\begin {itemize}
\item Write down the story on a white sheet labeled Press Release that exists for this purpose. Either you or the other player who wants to break the story can write it. 
\item Write down your internet alias as the person breaking the story.  
\item Roll d6.  You may add +/-1 to to the number you roll. This is your "Spin" number.
\item Deposit the press release in the envelope attached to the \sComputer{}. 
Tell a GM you have sent off a press release.
\end{itemize}

The GM then goes and gets the story from the packet, reviews the spin number, modifies a "Public Opinion" number that represents general public opinion on Heroes, and publicly posts the story on a "news board". 


This does not affect your ability to go through any HQ PR department if you declare for a side, you can go on doing both. 


Here are the instructions for the regular PR mechanic. 

Upholding the public image of the League of Heroes is of utmost importance to superheroes! Dragging the League's name through the mud is of high interest to all villains. Either way, gossip is power. The reputation of the League of Heroes is currently at a +5, which neither good nor bad. However, if it goes into the negatives the League is going have a PR crisis. 

Heroes or Villains who have declared for a side may call their HQ and ask their PR department to take a story to the press. This option is only available to characters who have declared for a team, but once you are declared there is no limitation on how many stories you can leak. There is also no limitation to what you can leak, but we ask that player exercise their common sense and keep it plausible. 

The player themselves have very minimal control over how the story breaks, but the way the story comes out will always benefit the Team (H/V) breaking the news. News will be broken in the order that the GMs receive stories. Even if the GMs receive the same story twice, both stories will be published. 

To break a story with HQ PR you do the following:
\begin {itemize}
\item Write down the story on a white sheet labeled Press Release that exists for this purpose. Do not be profligate! 
\item Write down your name as the person breaking the story. You may not lie. 
\item Roll a d6. Record this number as "Spin"
\item Deposit the press release in the envelope attached to the sign for your HQ (if you have one). If you aren't part of a team, you can't break a story unless you have another way to get information out. Tell a GM you have sent off a press release.
\end {itemize}
The GM then goes and gets the story from the packet, reviews the spin number, modifies a "Public Opinion" number that represents general public opinion on Heroes, and publicly posts the story on a "news board". 


\end{document}
