\documentclass[green]{LRSguildcamp1}
\begin{document}
\name{\gNewspaperPR{}}

As a reporter you can publish stories with your very own secret PR mechanic. It is very key that you keep your reporter status secret, do everything should be done discreetly. 

You may publish stories for other people once they have found out your identity. 

You may only publish stories that you believe to be true and have a source for. You may publish at maximum three articles per hour. You may publish at maximum a total of six articles  over the course of the entire game. 

%Here are the instructions for the regular PR mechanic. 

%Upholding the public image of the League of Heroes is of utmost importance to superheroes! 
%Dragging the League's name through the mud is of high interest to all villains. Either way, gossip is power. 

%The reputation of the League of Heroes is currently at a +5, which neither good nor bad. However, if it goes into the negatives the League is going have a PR crisis. 

%Heroes or Villains who have declared for a side may call their HQ and ask their PR department to take a story to the press. This option is only available to characters who have declared for a team, but once you are declared there is no limitation on how many stories you can leak. There is also no limitation to what you can leak, but we ask that player exercise their common sense and keep it plausible. 

%The player themselves have very minimal control over how the story breaks, but the way the story comes out will always benefit the Team (H/V) breaking the news. News will be broken in the order that the GMs receive stories. Even if the GMs receive the same story twice, both stories will be published. 

%To break a story with HQ PR you do the following:
%1. Write down the story on a white sheet labeled Press Release that exists for this purpose. Do not be profligate! 
%2. Write down your name as the person breaking the story. You may not lie. 
%3. Roll a d6. Record this number as "Spin"
%4. Deposit the press release in the envelope attached to the sign for your HQ (if you have one). If you aren't part of a team, you can't break a story unless you have another way to get information out. Tell a GM you have sent off a press release.
%5. The GM then goes and gets the story from the packet, reviews the spin number, modifies a "Public Opinion" number that represents general public opinion on Heroes, and publicly posts the story on a "news board".

As a freelance superhero, you currently cannot use the PR apparatus of the \cHeroLeague{} or the \cVillainCompact{}. However, should you decide to join one side or the other, you may continue to publish and use your chosen side's PR department simultaneously.  

To break a story you do the following:
\begin{enumerate}

\item Write down the story on a white sheet labeled Press Release that exists for this purpose. Please keep the story to around one paragraph. 

\item Write down Associated Press as the person breaking the story. You must indicate the name of the other character publishing the story if you are publishing this for someone else. 

\item Roll a d6. Record as the Spin. You may choose to modify this number by +/- 1; +1 benefits the heroes and -1 benefits the villains. Record this number as the Modifier. 

\item Deposit the \wPressRelease{} in the envelope attached to the \sEmails{} sign.

\item Tell a GM you have sent off a \wPressRelease{}.

\item The GM then goes and gets the story from the packet, reviews the Spin and Modifier and adjusts a "Public Opinion" number that represents general public opinion on Heroes, and publicly posts the story on a news board. 

\end{enumerate}


Keep in mind that you have been slowly whittling away at the \cHeroLeague{}'s image for a few years as part of your revenge for the Chicago Incident, but you're still not too sure if you want it to plummet into the negatives. That would affect many of the superheroes not on the Council and the life of \cYoungest{}.  Decide your moves accordingly.

\end{document}
