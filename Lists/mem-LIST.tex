%%%%%
%%
%% This file sets up the Mem, MemFold, and MemEnvelope datatypes, and
%% creates possible macros for each.
%%
%% The Mem datatype isn't really used directly; it's there so the
%% other datatypes can inherit and share its code.
%%
%%%%%

\DECLARESUBTYPE{Mem}{Element}
\PRESETS{Mem}{
  %% \MYname is the trigger
  \F\MYtext	%%  text
  }


%%%%%
%% MemFold and MemEnvelope are both subtypes of Mem.  MemFold is for
%% fold-n-staple style mempackets, MemEnvelope is for stuff-n-seal
%% style mempackets.  If you want a mempacket to contain interesting
%% contents, like sheets, abilities, and other mempackets, use a
%% MemEnvelope.
\DECLARESUBTYPE{MemFold}{Mem}
\DECLARESUBTYPE{MemEnvelope}{Mem}


%%%%%
%% MemCover and MemPage are for the cover and pages of mempacket
%% booklets, which resemble research notebooks.  These are good
%% substitutes for large piles of MemFolds, and can be useful for
%% things like amnesiac characters.
%%
%% Like MemFolds, MemPages shouldn't directly own any other elements
%% as contents.  Use MemEnvelope instead.
%%
%% MemPages are usually assigned to a MemCover (via \MYmems), with the
%% MemCover representing the entire booklet and assigned to a
%% character.
%%
%% A MemCover is not a mempacket in and of itself; its name is not its
%% trigger and its text is not a memory.
\DECLARESUBTYPE{MemCover}{Mem}
\PRESETS{MemCover}{
  \sd\MYtext	{Each page is a memory/event packet with a separate
		trigger.}
  }

\DECLARESUBTYPE{MemPage}{Mem}


%%%%%
%% \memfold{<trigger>}{<text>}
%% \memenvelope{<trigger>}{<text>}
%% \memcover{<name>}{<pages>}
%% \mempage{<trigger>}{<text>}
%% \startmembook{<name>} <pages> \endmembook
%%
%% These are wrappers around \INSTANCE, useful as 1-shots.
%% \startmembook...\endmembook is a simple wrapper around \memcover
%% that may have better syntax for use within character sheets (inside
%% a \starttag{mems}...\endtag block).
\newinstance{MemFold}{\memfold[2]}{
  \s\MYname{#1}\s\MYtext{#2}}
\newinstance{MemEnvelope}{\memenvelope[2]}{
  \s\MYname{#1}\s\MYtext{#2}}
\newinstance{MemCover}{\memcover[2]}{
  \s\MYname{#1}\s\MYmems{#2}}
\newinstance{MemPage}{\mempage[2]}{
  \s\MYname{#1}\s\MYtext{#2}}

\long\def\startmembook#1#2\endmembook{\memcover{#1}{#2}}


%%%%%%%%%%%%%%%%%%%%%%%%%%%%%%%%%%%%%%%%%%%%%%%%%%%%%%%%%%%%%%%%%%

% \NEW{MemFold}{\mTest}{
%  \s\MYname	{Test Trigger}
%  \s\MYtext	{This is a Test of a fold-n-staple memory packet}
%  }
  	
	 \NEW{MemFold}{\mOSTalkYS}{
  	\s\MYname	{Open if you talk one-on-one with a character with badge number 116.} %\cYS{\MYnumber}
  	\s\MYtext	{You realize that this is your former trainee \cYSOldName{} from the Chicago Incident. \cYS{\Theyare} much older, and difficult to recognize, but after hearing \cYS{\their} voice you are certain it's \cYS{\them}.  } 
		}
\NEW{MemFold}{\mOSLeaveLeague}{
  	\s\MYname	{Open if you leave the 
			League of Heroes.%\cHeroLeague{\intro}
			}
  	\s\MYtext	{If the League is in public favor, they start moving to reduce your influence in \pCityO{}.  Let $PR$ be the current PR score of the \cHeroLeague{\intro}.  If $PR \ge 5$, subtract $PR - 4$ from your Influence score.}
		}
		\NEW{MemFold}{\mTweenTalkYS}{
  	\s\MYname	{Open if you talk with a character with badge number 116 about their super powers. } % \cYS{} \cYS{\their}
  	\s\MYtext	{After talking with this person about \cYS{\their} super powers before long you realize that you remember another superhuman having those powers.  It might be a good idea to research them on \pTweenwebsite{}. } 
		}
\NEW{MemEnvelope}{\mTakeCity}{ % TODO: check with Acata that I did this right
	\s\MYname{Open if you take control of a city you attack.}
	\s\MYtext{You've gained at least temporary control of the city, but others may try to take it back.}
	\s\MYgreens{\gDefendCities{}}
}

\NEW{MemEnvelope}{\mLoseCity}{
	\s\MYname{Open if you lose control of a city.}
	\s\MYtext{Your allies hastily organize an attempt to retake your city.  You may follow the instructions to attack cities from the attached greensheet.  Your allies will be in place 20 minutes from now.  This will be your only chance, but if you succeed you will retake the city (as if it were your second successful attack).}
	\s\MYgreens{\gAttackCities{}}
}

\NEW{MemEnvelope}{\mBecomeHero}{
	\s\MYname{Open if you decide to become a hero, and have the endorsement of a League of Heroes member.}
	\s\MYtext{Use the \aMoralStand{} ability now.  At the end of your speech, announce ``I, {\em Your Name Here}, declare I am a superhero, with the name {\em Choose Superhero Name Here}!''  You are now a hero.  Announce whether or not you choose to join the \cHeroLeague{\intro} immediately.  If you do, you may wish to publish a press release about your decision, which would benefit the \cHeroLeague{}.  (Press release forms can be found at the \sPRHolder{} sign.)}
	\s\MYabils{\aMoralStand{}}
	\s\MYblues{\bHeroLeague{}}
	\s\MYgreens{\gPR{}}
}

\NEW{MemEnvelope}{\mBecomeVillain}{
	\s\MYname{Open if you decide to become a villain, and have the endorsement of a Villain Compact member.}
	\s\MYtext{Use the \aMonologue{} ability now.  At the end of your speech, announce ``I, {\em Your Name Here}, declare I am a supervillain, with the name {\em Choose Supervillain Name Here}!''  You are now a villain.  Announce whether or not you choose to join the \cVillainCompact{\intro} immediately.  If you do, you may wish to publish a press release about your decision, which would benefit the \cVillainCompact{}.  (Press release forms can be found at the \sPRHolder{} sign.)}
	\s\MYabils{\aMonologue{}}
	\s\MYblues{\bVillainCompact{}}
	\s\MYgreens{\gPR{}}
}


%%%%%%%%%%%%%%%%%%%%%%%%%%%%%%%%%%%%%%%%%%%%%%%%%%%%%%%%%%%%%%%%%%

