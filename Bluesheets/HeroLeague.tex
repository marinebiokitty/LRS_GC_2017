\documentclass[blue]{LRSguildcamp1}
\begin{document}
\name{\bHeroLeague{}}

\bigquote{``For the Greater Good''}

The \cHeroLeague{\intro} began as an alliance of heroes in the major cities. It ensures each city with a major villain is also countered with a capable hero. Heroes may take side-kicks, but only after the side-kick is fully vetted by the \cHeroLeague{}. 

The \cHeroLeague{} is controlled by a Council, members of which are nominated by fellow Council members and voted in by heroes in the \cHeroLeague{} each term. The Council is tasked to make the decisions for the the rest of the \cHeroLeague{}, mediate conflicts, and deal with Public Relations. The Council is composed of the President, Vice President, Recruitment Chair, Hero Educator, Risk Manager, Treasurer, House Manager, Social Chair, Secretary, Philanthropy and Community Service Chair. Good PR is the most important thing for the \cHeroLeague{} to maintain. This means that all members of the \cHeroLeague{} must stick to supporting any press releases the Council has published. Everything else goes in the \cHeroLeague{} Archive.

\begin{itemize}
\item The \cHeroLeague{} pays for your expenses if you file them correctly 
\item You are responsible for mentoring and recruiting a certain amount of heroes
\item Benefit from the \cHeroLeague{}'s experienced experienced support teams
\item Excellent side-kick referral service 
\item Awesome clubhouse-HQ
\item Great parties
\item Maintain a good face and present a united front
\end{itemize}

The \cHeroLeague{} is where all heroes have their start, with a suitable mentor, and is where they get their cities and assignments. There is no Junior League, although heroes who are minors are welcome to make their own teams. Minors can be part of the \cHeroLeague{} as long as they are sponsored by a full fledged member. New members are accepted in cohorts, around 3-6 people, and they are expected to work together and develop their powers with their cohort before being assigned their own city or official team. Full-fledged members take turns hosting cohorts and may mentor specific heroes if they wish but there is no expectation that they develop any kind of bond. Some popular full fledged heroes may have multiple cohorts each year.  

\end{document}
