\documentclass[char]{LRSguildcamp1}
\begin{document}
\name{\cGrad{}}
You are \cGrad{\intro}, the \cGrad{\offspring} of the \cOldest{\villain} \cOldest{\intro} and the \cOS{\hero} \cOS{\intro}.  Your parents have fought and bickered constantly since well before you were born, though you realized as you grew older that...honestly, they're just really attracted to each other.  They worked out an arrangement where they called a truce a few days a week to have a relationship, if you can call it that, and raise you.  The other part of the time, you were raised by a nanny.  You're not sure if your powers of invisibility were caused by your feelings of being overlooked as a child, or if it was the other way around -- if you accidentally went invisible so often your parents just couldn't see you. 

Your years off at \pSuperSchool{} and then college were a breath of fresh air: you were finally free of your parents' shadow and free of their constant bickering.  You realized the other students' home lives were much saner than yours: their parents fought, but not nearly as often; their parents cared.  It took the perspective of meeting other people to realize just how unhealthy your home life had been.  Finally on your own, you found yourself taking fashion design courses, and invited to join a group of students creating a startup to design superhero -- and perhaps supervillain -- costumes, designed to provide mobility, endurance, and other features useful to any superhuman.  Finally, you had a group of friends, people you shared goals with, people who understood you, a place you could contribute...

...until you graduated, and had to come home.  You couldn't afford to live anywhere else, not yet.  The startup hasn't taken off yet, not enough to help you find a place to live, and you weren't ready to ask your parents to fund your startup.  You've been living in the house they share when not fighting with each other for a few weeks, now, hating being home and hating being back around your parents' bickering again.  You've got a sewing machine in your room, and have spent hours on the phone with your college friends, but it's just not the same.

Your parents towed you along to this family reunion, which you grudgingly went along with.  But now that you're here, you realize that there are other people in the family who could help you come back to the startup and get things moving -- an endorsement, a prominent superhuman wearing one of your costumes, or even just initial investment could make the difference between the dream of a few college students and a blossoming career.

You've felt more enthused about your startup than about going into heroism or villainy -- either choice would probably lead to a shouting match with one of your parents, but you might need to pick a side to get an endorsement.  Your \cGrandma{\grandparent} \cGrandma{}, a supervillain going by \cGrandma{\MYsupername}, has mentioned to you -- very quietly -- that there may be some opportunities for you and your cousins to impress \cGrandma{\them} at this family reunion...and while \cGrandma{} \cGrandma{\themself} is a very classy dresser, \cGrandma{\their} minions' uniforms seem to be decidedly drab.  Alternately, your \cYoungest{\uncle} \cYoungest{} seems to be a prominent figure in the \cHeroLeague{\intro}; \cYoungest{\they} might be a useful investor.  You don't know \cYoungest{\them} especially well, though, because \cYoungest{\they} live all the way in \pCityYoungest{}. 

You spent some time with your cousins \cTeen{} and \cTween{} before going off to college, but you haven't seen them in several years.  They're pretty bright, though, and seem to have a pretty good relationship with their \cArchitect{\parent}, your \cArchitect{\uncle} \cArchitect{}.  But you know some things that they don't.

When you were much younger, \cArchitect{}'s \cAS{\spouse} \cAS{} brought \cTeen{} and \cTween{} to visit you in \pCityO{}, which was odd -- they had usually been quite close, and it was unusual for them to appear separately.  \cTeen{} was only six and \cTween{} was only one, so they didn't follow what was going on.  While your parents were out and about, \cAS{} stayed at your place, taking care of the three of you with the nanny.  \cArchitect{} came to visit, and \cArchitect{\they} and \cAS{} started shouting at each other.  They left to take the argument away from the kids, leaving you with the nanny, but you were curious, and slipped out from under your nanny's eyes to follow invisibly.  They came into a building \TODO{which?} together, and then -- you don't remember exactly what happened, but only flashes.  You saw a massive, clearly superhuman figure charge through the building.  You saw \cArchitect{} stand there, eyes wide, frozen, doing nothing, when \cArchitect{\their} power could have stopped the figure in their tracks. You saw the massive figure knock past \cAS{}.  You saw the rubble fall, crushing \cAS{}.  You saw \cArchitect{} run away.  And then you ran, from the sirens and rubble and collapsing building.  You don't know much about what happened after that, other than the official reports (see \bChicagoIncident{}); you ran home to hide in your room.  You were just a kid, after all.
  You have never told your cousins -- you know it would hurt them badly.  But now that you're older and back home, maybe it's time you learned more about what happened. \TODO{needs more motivation?  Maybe has heard of YS' presence?}

Finally: your youngest cousin \cTween{} is trying to decide whether to go to \pSuperSchool{} or \pNormalSchool{}.  You enjoyed \pSuperSchool{} well enough, but when you really came into your own was when you stopped trying to be a superhero or a supervillain and started working on a business.  \cTween{}'s powers are super hearing, and \cTween{\theyare} interested in blogging and gathering news.  You think \cTween{\they} would be better off at \pNormalSchool{}, figuring out how to turn their skill into a business.  Super hearing isn't an especially useful power for a hero or a villain, so why not find another way to use it?

\begin{itemz}[Goals]
	\item Get your super costuming startup off the ground, potentially getting you an in with some major customers.
	\item Impress \cGrandma{} by accomplishing her tasks.
	\item Find out the truth of the \pCityO{} Incident.
	\item Convince \cTween{} to go to \pNormalSchool{}.

\end{itemz}

\begin{itemz}[Notes]
	\item \cGrandma{} also mentioned that you should start by finding \cGrandma{\their} lair.   
\end{itemz}

\begin{contacts}
	\contact{\cOldest{\intro}} (super name \cOldest{\MYsupername}) Your villainous \cOldest{\parent}, who spent your childhood more preoccupied with \cOS{} than you.
	\contact{\cOS{\intro}} (super name \cOS{\MYsupername}) Your heroic \cOS{\parent}, who spent your childhood more preoccupied with \cOldest{} than you.
	\contact{\cArchitect{\intro}} Your \cArchitect{\uncle}, a superpower nullifier who lost \cArchitect{\their} \cAS{\spouse} in the \pCityO{} Incident.
	\contact{\cYoungest{\intro}}  (super name \cYoungest{\MYsupername}) Your \cYoungest{\uncle}, a major member of the League of Heroes based in \pCityYoungest{}.
	\contact{\cGrandma{\intro}}  (super name \cGrandma{\MYsupername}) Your \cGrandma{\grandparent}, an infamous \cGrandma{\villain}.
	\contact{\cTeen{\intro}} Your cousin, the older \cTeen{\offspring} of \cArchitect{\Uncle} \cArchitect{}.
	\contact{\cTween{\intro}} Your cousin, the younger \cTween{\offspring} of \cArchitect{\Uncle} \cArchitect{}.
\end{contacts}

\end{document}
