\documentclass[char]{LRSguildcamp1}
\begin{document}
\name{\cYoungest{}}

You are \cYoungest{\MYsupername}. People often ask if you're an only child. You can be a bit...spoiled, but that happens when you're an awesome superhero adored by masses of people. Part of the confusion may also come from the fact that neither of your two siblings \cOldest{} and \cArchitect{} are heroes. It is complicated.  

You were raised by \cGrandma{\MYsupername}, a supervillainess, who allowed you as much freedom as you wanted. \cGrandma{} taught you how to manipulate all kinds of locks and all kinds of people. You have felt guilty at times that you didn't follow in her footsteps and that you chose your friends over your family, but you know that she is proud of your choice. 

With a parent like that, how did you end up choosing the path of heroism? It happened like this. Four years in \pSuperSchool{} were some of the best years of your life. You thrived in an environment where you were able to make new friends and impress your teachers. You and your best friends formed The Super 5! to fight petty crime in the streets and get dates for prom. Unexpectedly, the group blew up and you did a lot of good. You fed off the adoration of the public and the way people called you a hero. It was what you always imagined your life to be like.  
 
After The Super 5! disbanded, you went on through the \cHeroLeague{\intro} program and eventually joined the Heroes as a full-fledged member. Recently, you've even risen to a position on the Council, Social Chair. This is something that you owe to all the excellent PR stories the \cHeroLeague{} has been able to tell about your exploits. You should be happy. 

But you aren't. You know that you are the poster child for going good. Everyone on the Council says that you did right thing by deciding to break with your super villain relatives. That makes it even harder for you to walk this tightrope and keep up this farce. Is it such a terrible thing that you love your family? 
You wish you could tell everyone how you really feel, but your teammates are always prattling on about PR. 

Recently the spate of press exposures on scandals in the hero community has shaken you to your very core. You've had to play the media game even more to keep the \cHeroLeague{}'s public face from sliding into the abyss of no return. What will happen if the press discredits the \cHeroLeague{}? If there is no \cHeroLeague{}, then you won't have any friends and you won't be loved anymore, what will you do? This is what concerns you the most. 

This is probably the reason why you have started thinking about getting married. Two years ago, when you took it upon yourself to mentor some of the more troubled members in the \cHeroLeague{} you met \cYS{} and fell head over cape. In the groups you ran, \cYS{\they} was always quiet and observant little presence in the back of the room. At first, you despaired of ever getting \cYS{\them} to open up. Until, \cYS{\they} got in a fight one session about the Chicago Incident. The Chicago Incident was one of the worst in superhero history, a blemish upon the \cHeroLeague{} that continues to affect the emotions of many who had lost someone in the carnage. \cArchitect{} had lost their spouse \cAS{\intro}. They really loved each other, and you're not sure if \cArchitect{\they} has ever truly gotten over it. 

After the fight, you started to have one-on-one counseling sessions with \cYS{} and even though it took some time, you feel like you finally got through to them. \cYS{} seemed to be happier in the \cHeroLeague{} and there were less conflicts.  All \cYS{\they} really needed was someone to try and reach out to them. Things just took their natural course. Even though \cYS{\they} is a person that you thought about every da , you wish you understood \cYS{\them} better though. Maybe if you figured out what happened to \cYS{\them} you might be able to help them move past it.  
 
You've spent many afternoons staring mindlessly out from your fancy \cHeroLeague{} office contemplating true love and happy endings. It might be what is missing in your life. You've psyched yourself up to propose, but as you were thinking over how to plan the proposal you realized that the proposal wouldn't be perfect unless you could do it with the engagement rings. Way back when \cGS{\parent} proposed to \cGrandma{\parent}, he challenged her to complete a heist on the most expensive jeweler in town. When \cGrandma{\they} opened the box, she saw the rings that he had commissioned, two square cut 8 carat milochite stones, rare gems that enabled the wearer to communicate with each other over long distances. The rings are one of the only possessions of \cGS{\parent} that remain after \cGS{\their} \cGrandma{\they} hasn't worn those stones for a long time and you are entitled to them as her favorite child.  

 \cGrandma{\parent} surprised you when \cGrandma{\they} called this family dinner, but you are happy because this is a perfect opportunity to introduce \cYS{} to the rest of the family. You aren't sure what they will think. You really want \cGrandma{\parent} to approve of \cYS{}. 

This dinner is also good time to get some tasks done. Firstly, the \cHeroLeague{} has tasked you with convincing \cArchitect{} to build the new East Coast HQ. It'll be a grand affair with a coffee bar and maybe even a massage room. You're going to be competing with \cOldest{} who wants \cArchitect{} build the villain HQ, there's no way that \cArchitect{} will be persuaded to do that! You're confident in your powers of persuasion and the powers of the money you have to convince \cArchitect{}. 

The less pleasant task is that you've been formally asked to pass on the message to \cGrandma{\parent} to be good. Your mother is a force of nature, but the superhero on duty in \pCityGrandma{} is never able to last out the year. Convincing her to be good is the lesser of two evils because you don't want to be reassigned to be the hero to counter your mother if they run out of candidates. 

There's an opportunity at this dinner to mentor some prospective heroes. You've always enjoyed the teaching role. Your nieces and nephews are coming up on important decisions in their lives and you want to help guide them. It's past time for \cGrad{} and \cTeen{} to join the fold. You think that \cTween{} will have a great experience at \pSuperSchool{} because in your opinion \pSuperSchool{} is the best place on this earth for budding supers.  Maybe \cTween{} will be able to start their own super team like The Super 5!.The application deadline is tonight, and you think that your opinion is really going make a difference. 

With such an important occasion, and since you're going to be gone from your city for an entire evening, you've hired some city-sitters and back-up. Luckily you're a popular person to side-kick or city-sit for, which is why nothing bad is going to happy while you're away. You've left the details to the \cHeroLeague{}, they will make everything right. 

\begin{itemz}[Goals]
	\item Keep the PR score of the \cHeroLeague{} as high as possible.
	\item Try to make \cGrandma{} be good without offending her, perhaps with some form of bribery, emotional or monetary. 
	\item Obtain the \iEngagementRings{}, by persuading \cGrandma{} to give them to you. If persuasion really doesn't work, then think about taking them. 
	\item After obtaining the rings, decide whether or not to propose. Does \cYS{} love you?  
	\item Find out what has been bothering \cYS{}. 
	\item You think you have got it locked down with \cArchitect{} to design the new Superhero HQ, but just need confirmation. You are willing to pay a significant amount of money to do this.  
	\item  Recruit some more superheroes and prevent \cOldest{} from recruiting new supervillains.
	\item Convince Tween to go to \pSuperSchool{}. 
	
\end{itemz}

\begin{itemz}[Notes]

	\item While you like just about everyone in your family, even \cOldest{}, you're quite competitive. You're also used to having things come easily to you and winning. You're nice but not THAT nice. 
	\item You grew up with \cGrandma{}.  Just because you're in the \cHeroLeague{} now doesn't mean that you don't steal or eavesdrop. 
	\item You understand that this is a huge step for \cYS{}, and since \cYS{\they} \cYS{\have} never talked much about family, you're going to do your best to make \cYS{\them} comfortable. 
	\item The \cHeroLeague{\intro} has freed up a significant amount of cash for your usage tonight, mostly to help convince \cArchitect{} to build the hero HQ.  You may find other uses arise over the course of the evening, however.
	
\end{itemz}

\begin{contacts}
	\contact{\cYS{}}  Your lover and hopefully future spouse, who you don't always understand.
 	\contact{\cOldest{}} Your oldest sibling. A bland sort of \cOldest{\villain}. 
	\contact{\cOS{}} The \cOS{\hero} that has been in a torrid romance with \cOldest{} for twenty-two years.
	\contact{\cGrandma{}} Your \cGrandma{\parent}, \cGrandma{\villain} extraordinaire.  You are \cGrandma{\their} favorite child and you love \cGrandma{\them}
	\contact{\cArchitect{}} Your middle sibling. It's sad how \cArchitect{\their} powers have made others wary of them. You feel it too, but you need them to accept the Villain Compact's bid for a new HQ.
	\contact{\cGrad{}} Your wayward \cGrad{}, who you don't know well, but could be hero material.
	\contact{\cTeen{}} \cArchitect{}'s older child, a technical wizard and potential hero. 
	\contact{\cTween{}} \cArchitect{}'s younger child, has super-hearing and is quite rambunctious. Kind of reminds you of yourself as a kid. Needs a bit of discipline or a good mentor to straighten them out. 
	
	. 
	
\end{contacts}

\end{document}
