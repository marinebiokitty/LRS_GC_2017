\documentclass[char]{LRSguildcamp1}
\begin{document}
\name{\cTween{}}

\bigquote{``I'm not a conspiracy theorist - I'm a conspiracy analyst.''}{- Gore Vidal}
\bigquote{``When you have eliminated the impossible, whatever remains, however improbable, must be the truth.''}{- Sherlock Holmes}

Your name is \cTween{\intro} and you are 13 and in 8th grade. You are a connoisseur with a refined taste when it comes to the subtle truths of the world. Your book cases are lined with Ripley Believe-it-or-not dvds, aliens sighting documentaries,  BBC Sherlock, and the collector's edition for the Hunt for Mythical Creatures. The chapter on Nessie still warms your heart. You have tried to describe to your \cArchitect{\parent}, \cArchitect{}, and your older \cTeen{\sibling}, \cTeen{} how deeply important finding out and sharing the truth is, but but they both seemed to think you were a bit obsessed and it was a passing phase.
You disagree. Your not sure what your \cAS{\parent}, \cAS{}, would have thought about your interests. \cAS{\they} died when you were only a year old in the \pCityO{} Incident. You abstractly miss your \cAS{\parent}, but it hard to miss what you never knew. Your \cArchitect{\parent} tells you stories when you ask, but does not bring \cAS{\them} up much. \cTeen{} gets really quiet and introspective when the topic comes up. 

Your power is \cTween{\MYsuperpower} and has served you well in getting to the bottom of mysteries. A Super power alone does not bring answers, careful planning,deduction, and a willingness to make leaps of imagination are necessary. That being said, having a Super power is really useful. You figured out who nicked the school dance money in 4th grade and which teacher hit the principal's car in 6th grade. You always get to the bottom of mysteries, sometimes it takes a few steps. For example, you got the teacher who hit the principal's car wrong twice, but the third one was spot on. You can't make a cake without breaking a few eggs. Your school and \cArchitect{\parent} were not amused but \cGrandma{} and your older \cOldest{\uncle}, \cOldest{}, were greatly. \cYoungest{}. Your younger \cYoungest{\uncle}, \cYoungest{}, took it in good humor and applaud your detective skills.

For the last two years, you have taken your sleuth work online to the \pTweenwebsite{} website, which is the hub for all fan rumors and stories about Supers. Your screen name is The Ear. You have spent a lot of time raising your level of being \textsl{InTheKnow}. This ranking equates to how much creditability your have on the site, you currently are at the level of \textsl{solidly InTheKnow}. You have a fairly large following and your word carries weight. The next and highest level is \textsl{epically InTheKnow}. You have gathered questions together that people want to know about your super family. If you can get information on all of these questions you can level up. As a member with \textit{Epically InTheKnow} status, you would get unfettered access to all the forums, your answers would always appear on the top of forms, and you could begin to start mentoring others on the site \pTweenwebsite{}. You have chosen questions that are interesting but innocuous as you would not betray your family. If the older members of your family found out, however, they would not be happy you were sharing any kind of secrets with the internet. There would be consequences. 

What is important in your life right now seems to be up for debate. Your family members keep talking about what high school you should go to, but from your point of view, they all seem to be missing the point. For you, the choice is not between a super or non-super school or even being  between hero or villain, but rather which place can get you the best angle to hone your skills! If you went to the a non-super school, like \pNormalSchool{}, you could keep your power a secret and use it with other none the wiser. You could continue to build your cred on the internet and really launch out in college. If you went to the super school, \pSuperSchool{}, you would be near the source of so many of the mysteries the internet wanted solved. That level of access would be unprecedented, but your online identity might discovered and outed. Students at \pSuperSchool{} would not like someone reporting on their lives and have the resources to shut you down. 
The decision \textsl{must be turned in tomorrow}, so \textit{you need to make up you mind tonight!}

\cGrandma{} has called a family meeting in \pCityGrandma{} and you were not so psyched to go until you got an message from \cGrandma{}. \cGrandma{\they} has promised to be holding a secret competition the grandkids with several prizes that would be of interest. \cGrandma{} has never been one to jest about cool gifts. You are intrigued.  

Your extended family will be at there as well. \cOldest{}, your \cOldest{\uncle}, who is a villain and his lover, \cOS{}, who is a hero are coming. You cannot say their relationship is loving but has been going on for a long time. No one is allowed to mess with their city but them. They are a source of a lot of speculation on the internet, everything from fan fic to burning celebrity questions. Thus you are always happy to see them. \cTeen{} would often go spend summer with their child, \cGrad{}, before \cGrad{\they} went off to college.  \cYoungest{}, your youngish \cYoungest{\uncle} will also be there. \cYoungest{They} are a hero in \pCityYoungest{}, you spent some time with \cYoungest{\them} last summer and had a great time going to several cons.  \cYoungest{} has been dating someone new a \cAS{\hero} named \cAS{} in \pCityYoungest{}. 


\begin{itemz}[Goals]
	\item \textit{Find out and publish} on \pTweenwebsite{} the answers for all questions in your notebook about your Super family that users on \pTweenwebsite{} want to know to gain textit{Epically IntheKnow} status. See Note Book.	After every 3 questions you publish, you will receive 10,000 from the GM.
	
	\item Find out more about \cAS{}. This could be potentially a new source to publish about on \pTweenwebsite{}.
	
	\item Choose whether to go to \pSuperSchool{} or \pNormalSchool{}
	
	\item Win prizes from \cGrandma{}'s scavenger hunt and beat your cousin and sibling
\end{itemz}

\begin{itemz}[Notes]
	\item 
\end{itemz}
\begin{contacts}
	\contact{\cGrandma{}} This is your Grandma and is a renowned villain, art thief, and occasional high profile celebrity kidnapper. Her house has lots of cool hidden parts and traps!
	
	\contact{\cOldest{}} This is your \cOldest{\uncle} and is a villain. 
	
	\contact{\cOS{}} This is \cOldest{}'s spouse and a hero.You are interested in how a hero and villain being in a relationship work. 
	
	\contact{\cGrad{}} \cGrad{} is \cOldest{}'s and \cOS{}'s child and your only cousin. \cGrad{\They} just graduated from college, but has not made a move for hero or villain yet. 
	
	\contact{\cYoungest{}} \cYoungest{} is your younger \cYoungest{\uncle} and a \cYoungest{\hero}. \cYoungest{} lives in \pCityYoungest{} and you visited \cYoungest{\them} last summer.  
	
	\contact{\cYS{}} \cYS{} is a hero and has apparently been dating \cYoungest{}. You don't know much about \cYS{\them}, but you are going to change that. 
	
	\contact{\cArchitect{}} Your \cArchitect{\parent} who works as an architect in a powerful firm in \pCityArchitect{}.  You love your \cArchitect{\parent}, but feel like \cArchitect{\they} don't get you. Your \cArchitect{\parent} is the only person in your family who 
	
	 	\contact{\cTeen{}} Your older \cTeen{\sibling} is a high school student at \pNormalSchool{} and loves to build tech. You hate it when \cTeen{} tests new things on your stuff. 
		
\end{contacts}

\end{document}
