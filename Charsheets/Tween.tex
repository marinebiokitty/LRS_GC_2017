\documentclass[char]{LRSguildcamp1}
\begin{document}
\name{\cTween{}}

\bigquote{``I'm not a conspiracy theorist - I'm a conspiracy analyst.''}{- Gore Vidal}


\bigquote{``When you have eliminated the impossible, whatever remains, however improbable, must be the truth.''}{- Sherlock Holmes}


Your name is \cTween{\intro} and you are 13 in 8th grade. You are a connoisseur with a refined taste when it comes to the subtle truths of the world. Your book cases are lined with Ripley Believe-it-or-not dvds, aliens sighting documentaries, and BBC Sherlock, and the collector's edition for the hunt for mythical creatures, the chapter on Nessie still warms your heart. You once tried to describe to your \cArchitect{\parent}, \cArchitect, and your older c\Teen{\sibling}, c\Teen{} how deeply you felt connected to finding out and sharing the truth, but but they both seemed to think you like stories a bit too much and it was a phase. You disagree.


You know power alone did not bring you the answers, but careful planning and deduction were necessary and a willingness to make leaps of imagination. You often discuss your current cases with the poster of Sherlock on your wall. That being said, having a super power is really useful. Your power is super hearing and has served you well in getting to the bottom of mysteries, like who had nicked the school dance money in 5th grade and which teacher knocked over the science experiments 7th grade while making out. You always get to the bottom of mysteries, sometimes it takes a few steps. For example, you got the pair of teachers wrong twice, but the third one was spot on. You can't make a cake without breaking a few eggs. Your school and \cArchitect{\parent} were not amused but c\Grandma{} and your c\Oldest{\uncle}, \cOldest{} were greatly.

You know power alone did not bring you the answers, but careful planning and deduction were necessary and a willingness to make leaps of imagination. You often discuss your current cases with the poster of Sherlock on your wall. 




What is important in your life right now seems to be up for debate. Your family members keep talking about what high school you should go to and while you do have opinions, they all seem to be missing the point. The \textbf{really} important point here is your latest find! 

\begin{itemz}[Goals]
	\item 
\end{itemz}

\begin{itemz}[Notes]
	\item 
\end{itemz}

\begin{contacts}
	\contact{} This is the format for contacts 
\end{contacts}

\end{document}
