\documentclass[char]{LRSguildcamp1}
\begin{document}
\name{\cTween{}}

\bigquote{``I'm not a conspiracy theorist - I'm a conspiracy analyst.''}{- Gore Vidal}
\bigquote{``When you have eliminated the impossible, whatever remains, however improbable, must be the truth.''}{- Sherlock Holmes}


Your name is \cTween{\intro} and you are 13 and in 8th grade. You are a connoisseur with a refined taste when it comes to the subtle truths of the world. Your book cases are lined with Ripley Believe-it-or-not dvds, aliens sighting documentaries,  BBC Sherlock, and the collector's edition for the hunt for mythical creatures. The chapter on Nessie still warms your heart. You once tried to describe to your \cArchitect{\parent}, \cArchitect{}, and your older \cTeen{\sibling}, \cTeen{} how deeply you felt connected to finding out and sharing the truth, but but they both seemed to think you like stories a bit too much and it was a phase. You disagree. Your not sure what your \cAS{\parent} would have thought about it as \cAS{\they} died when you were only a year old in the \pCityO{} Incident. The rest of your family is seems sadder about this than you, but you never met \cAS{\them} so its hard to say. You miss the potential, but have decide to protect and nurture what you have instead.


You know power alone did not bring you the answers, but careful planning and deduction are necessary and a willingness to make leaps of imagination. That being said, having a super power is really useful. Your power is super hearing and has served you well in getting to the bottom of mysteries, like who had nicked the school dance money in 5th grade and which teacher hit the principal's car in 7th grade . You always get to the bottom of mysteries, sometimes it takes a few steps. For example, you got the teacher who hit the principal's car wrong twice, but the third one was spot on. You can't make a cake without breaking a few eggs. Your school and \cArchitect{\parent} were not amused but \cGrandma{} and your older \cOldest{\uncle}, \cOldest{}, were greatly. \cYoungest{}. Your younger \cYoungest{/uncle}, \cYoungest{}, took it in good humor and applaud your detective skills.


What is important in your life right now seems to be up for debate. Your family members keep talking about what high school you should go to and while you do have opinions, they all seem to be missing the point. The choice is not between super and non-super schools or even hero and villain, but where you can get the best angle to hone your skills! If you went to the non-super school, \pNormalSchool{}, you could keep your power a secret and use it with other none the wiser. If you went to the super school, \pSuperSchool{}, you would be near the source of so many of the mysteries the internet wanted solved. That level of access would be great, but being at \pNormalSchool{} would give you more chance to hone skills and build credibility. Its a tough decision. The decision \textsl{must be made by tomorrow}, so \textit{you need to make up you mind tonight!}
One of your latest hobbies have been trolling the internet to see what fan theories about your super family are. You believe the fans have a right to know and its a challenge to find out. You, of course, choose innocuous fact as you won't betray your family trust too much. You have promised the fan site, \pTweenwebsite{}, confirmation and proof if possible. Those who provider information for the \pTweenwebsite{} are referred to as InTheKnow and receive different levels.  For every rumor you can comfirm you receive 50 more points and if you have proof it goes up by 200. If you can get 800 more points, you will unlock the next level of Epically InTheKnow. As a member with Epically InTheKnow in the known status, you would get unfettered access to all the forums, your answers would always appear on the top of forms, and you could begin to start mentoring others on the site \pTweenwebsite{}.

Though your \cYoungest{\uncle}, \cYoungest{}, has been dating someone new a \cAS{\hero} named \cAS{} in \pCityYoungest{}. You briefly met \cAS{} during a summer visit last year where you over heard a phone call \cAS{\they} was on talking about deadlines and stories. \cAS{} presents as a \cAS{\hero} but the conversation sure sounded something like a reporter would say. You promised yourself the next time you were in the same place you would get to the bottom of this and find out what \cAS{\their} intentions were for your family.

\cGrandma{} has called a family meeting in \pCityGrandma{} and you were not so psyched to go until you got an message from \cGrandma{}. \cGrandma{\they} have promised to be holding a secret competition the grandkids with several prizes that would be of interest. \cGrandma{} has never been one to jest about cool gifts. You are intrigued.  

Your extended family will be at there as well. \cOldest{}, your \cOldest{\uncle}, who is a villain and his lover, \cOS{}, who is a hero. You cannot say their relationship is loving but has been going on for a long time. No one is allowed to mess with their city but them. They are a source of a lot of speculation on the internet, everything from fan fic to burning celebrity questions. \cTeen{} would often go spend summer with their child, \cGrad{}, before \cGrad{\they} went off to college.  \cYoungest{}, your youngish \cYoungest{\uncle} will also be there. \cYoungest{They} are a hero in \pCityYoungest{}, you spent some time with \cYoungest{\them} last summer and had a great time going to several cons.  



\begin{itemz}[Goals]
	\item Fill out your notebook with thriva about your super family the fan on the internet want to know
	\item Find out what \cAS{} is really up to and \cAS{their} intentions. You want to gather some proof, but to not blow the whistle without cause.
	\item Choose what high school to go to 
\end{itemz}

\begin{itemz}[Notes]
	\item 
\end{itemz}
\begin{contacts}
	\contact{\cGrandma{}} This is your Grandma and is a renowned villain, art thief, and occasional high profile celebrity kidnapper. Her house has lots of cool hidden parts and traps!
	\contact{\cOldest{}} This is your  \cOldest{\uncle} and is a villain. 
	\contact{\cOS{}} This is \cOldest{}'s spouse and you are interested in how a hero and villain being married dynamics works. \cOS{\uncle}.
	\contact{\cGrad{}} \cGrad{} is \cOldest{}'s child and your only cousin who just graduated from college. \cGrad{} is rumored on the internet to have floated some clothing designs around but has been keeping it on the secret from everyone.
	\contact{\cYoungest{}} \cYoungest{} is your uncle and \cYoungest{\hero}. \cYoungest{} lives in \pCityYoungest{} and you visited \cYoungest{\them} last summer.  
	\contact{\cYS{}} \cYS{} is a hero and apparently been dating \cYoungest{} and you met last summer. You overheard a conversation \cYS{} was having and have suspicions about \cYS{them} now.
	\contact{\cArchitect{}} Your \cArchitect{\parent} who works as an architect in a powerful firm in \pCityArchitect{}.  You love your \cArchitect{\parent}, but feel like you two need something new to bond over.
	 	\contact{\cTeen{}} Your younger \cTeen{\sibling} is a high school student at \pNormalSchool{} and loves to build tech. You hate it when \cTeen{} tests new things on your stuff, like the heat ray. 
\end{contacts}

\end{document}
