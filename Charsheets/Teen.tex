\documentclass[char]{LRSguildcamp1}
\begin{document}
\name{\cTeen{}}

\bigquote{``We don't want tradition. We want to live in the present and the only history that is worth a tinker's dam is the history we make today.''}{- Henry Ford}

\bigquote{``To invent you need a good imagination and a pile of junk.''}{- Thomas Edison}

Your name is \cTeen{} and you are 16 years old. The best part about your life is your power. You have an innate knowledge of how to build gadgets and super-tech. Your skill grows with ever passing project and some of your latest triumph have included an ice and heat ray. Your \cOldest{\uncle} help provide the necessary starting material for your ice ray and it was not too hard to create heat ray afterward. 

You attend \pNormalSchool{} a non-super high school in \pCityArchitect{}, which actually is mostly ok for you. You have friends and enjoy being the center of attention. You \textsl{could} be the top of your classes if you choose, but really don't see the point. As long as, this class work does not get in the way of your actual work with building cool stuff its all fine. Another reason you like your school is that you feel like the ruler of a very small realm here, which admittedly is nice. The other students cannot match the pace at which you make your gadgets. You became a local legend last week by freezing an annoying sub during class. It was worth the suspension and the lecture by your parent, \cArchitect{}. Though, your heat and freeze rays have been taken though and that sucks. 

However your sights are set much higher. You want to live as a super. You are ok not attending \pSuperSchool{} because you are afraid your talent would not be recognized and you would be slotted into the sidekick track and not the main super track. You are \textbf{not} a sidekick. You have been biding your time and honing your skills so you can debut as a super while in college. In private, you have sketch out what your costume and theme music might be. Its good to be prepared. You have even dressed up in nondescript clothes to try and pull some pranks.  They went mostly fine -- your gadgets only sometimes blow up and only minor singeing.  It is all in the service of learning. If you don't push yourself and tools into ever more extreme and varied situations, how can you be prepared. You are angling to get some more road tests for inventions next break. Your anuts and uncles all hold cities, if only you could convince one of them to let you city-sity for a while.

\cGrandma{} is your grandma and is a famous \cGrandma{\villain}. You think she is pretty cool. Your \cArchitect{\parent} is alright but is mostly at the office drawing up some kind of blueprints for a building. When you were younger, the two of you got along well, but that was before the death of your \cAS{\parent}, \cAS{}, in the \pCityO{} incident. Before that incident, you would go to your \cArchitect{\parent} work and spend hours looking at plans for buildings, but when you interests turned towards tech something changed. Your \cArchitect{\parent} provides for you in the terms of work space and hack-a-thon style camps, but keeps not so subtlety trying to steer your interests towards the mundane world building and tech firms. Your younger \cTween{\sibling}, \cTween{}, is three years younger then you and undeniably a tween. \cTween{}'s a cool kid with a taste for ferreting out secrets, but has a wild imaginations when it comes to conspiracy theories. \cTween{\they} are more often than not wrong, but just sometimes hits on a deep truth. \cTween{} is going to high school soon and will either go to \pNormalSchool{} or \pSuperSchool{}. You are not sure if you want to share your kingsom with \cTween{\him} or not, but it could be useful to have a potential ally good at getting information. 

In school, you have recently read about the Chicago Incident in class and while most of the world knows this is where some super hero trainees got carried away while heroing and damage the city. For you, this is the incident where your \cAS{\parent} died. The details surrounding this are vague and you no longer feel satisfied with the story you were told as a child. That it was just a terrible quincidence, your \AS{} was a hero and heroes just don't go out like that. You sometimes feel distance from your\cArchitect{\parent} and \cTween{} about your \cAS{\parent}. Your \cArchitect{\parent} does not like to talk about \cAS{} and \cTween{} was only one and does not remember. You are not sure you want to involve either in your search.

\cGrandma{} has called a family meeting in \pCityGrandma{}

 
\begin{itemz}[Goals]
	\item 
\end{itemz}

\begin{itemz}[Notes]
	\item 
\end{itemz}

\begin{contacts}
	\contact{}  This is the format for contacts 
\end{contacts}

\end{document}
