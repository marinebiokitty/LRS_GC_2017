\documentclass[char]{LRSguildcamp1}
\begin{document}
\name{\cTeen{}}

\bigquote{``We don't want tradition. We want to live in the present and the only history that is worth a tinker's dam is the history we make today.''}{- Henry Ford}

\bigquote{``To invent you need a good imagination and a pile of junk.''}{- Thomas Edison}

Your name is \cTeen{\intro} and you are 16 years old. The best part about your life is your power. You have an innate knowledge of how to build gadgets and super-tech. Your skill grows with every passing project and some of your latest triumphs have included ice and heat rays. Your \cOldest{\uncle} helped provide the necessary starting material for your ice ray and it was not too hard to create heat ray afterward. 

You attend \pNormalSchool{}, a non-super high school, in \pCityArchitect{}, which actually is mostly ok for you. You have friends and enjoy being the center of attention. You \textsl{could} be the top of your classes if you choose, but really don't see the point. As long as, this class work does not get in the way of your actual work of building cool stuff its all the same to you. Another reason you like your school is that you feel like the ruler of a very small realm here, which admittedly is nice. The other students cannot match the pace at which you make your gadgets. You became a local legend a two weeks ago by freezing an annoying substitute teacher during class. It was worth the current suspension and the lecture by your parent, \cArchitect{}. Though, your heat and freeze rays have been taken though and that sucks. 

However your sights are set much higher. You want to live as a super. You are ok not attending \pSuperSchool{} because you are afraid your talent would not be recognized and you would be slotted into the sidekick track and not the main super track. You are \textbf{not} a sidekick. You have been biding your time and honing your skills so you can debut as a super while in college. In private, you have sketch out what your costume and theme music might be. Its good to be prepared. You have even dressed up in nondescript clothes to try and pull some pranks.  They went mostly fine -- your gadgets only sometimes blow up and only minor explosions. It is all in the service of learning. If you don't push yourself into ever more extreme and varied situations, how can you be prepared? You are angling to get some more road tests for inventions next break. Your anuts and uncles all hold cities, if only you could convince one of them to let you city-sit for a while.

\cGrandma{} is a famous \cGrandma{\villain}. You think she is pretty cool. Your \cArchitect{\parent} is alright but is mostly at the office all day. When you were younger, the two of you got along well. You would go to your \cArchitect{\parent} work and spend hours looking at plans for buildings, but when your interests turned towards tech, something changed. Your \cArchitect{\parent} provides for you in the terms of work space and hack-a-thon style camps, but keeps not-so-subtlety trying to steer your interests towards the mundane world building and tech firms. Your younger \cTween{\sibling}, \cTween{}, is three years younger than you and undeniably a tween. \cTween{}'s a cool kid with a taste for ferreting out secrets, but has a wild imaginations when it comes to conspiracy theories. \cTween{\They} are more often than not wrong, but just sometimes hits on a deep truth. \cTween{} is going to high school soon and will either go to \pNormalSchool{} or \pSuperSchool{}. You are not sure if you want to share your kingdom with \cTween{\them} or not, but it could be useful to have a potential ally good at getting information. One major hold back is when your were at the same school while younger, \cTween{} would embarrass you by spouting ridiculous rumors loudly in the playground and you lost some social standing because of this. \cTween{} is older now and this might not happen again, but you can't be sure. \cTween{} must submit the paperwork tomorrow.

In school, you have recently read about the Chicago Incident in history class and while most of the world knows this is where some super hero trainees got carried away while heroing and damage the city. For you, this is the incident where your \cAS{\parent} died. The details surrounding this are vague and you no longer feel satisfied with the story you were told as a child. That it was just a terrible quincidence, your \cAS{} was a hero and heroes just don't go out like that. You sometimes feel distance from your\cArchitect{\parent} and \cTween{} about your \cAS{\parent}. Your \cArchitect{\parent} does not like to talk about \cAS{} and \cTween{} was only one and does not remember. You are not sure you want to involve either in your search for the truth.

\cGrandma{} has called a family meeting in \pCityGrandma{} and you were not so psyched to go until you got an message from \cGrandma{}. \cGrandma{\they} have promised to be holding a secret competition for your and your cousin with several prizes that would be of interest. \cGrandma{} has never been one to jest about cool gifts. You are intrigued.  Your extended family will be at there as well. \cOldest{}, your \cOldest{\uncle}, who is a villain and his lover, \cOS{}, who is a hero. 
You got along better with \cOS{} than \cOldest{} over the years, \cOS{\they} was nicer and electiry and tinkering go hand-in-hand. Furthermore, it has been good for your rep to be related to this super couple as they are popular as a hero/villain odd pairing. You cannot say their relationship is loving but has been going on for a long time. No one is allowed to mess with their city but them.  You would often go spend summer with their child, \cGrad{}, before \cGrad{\they} went off to college. Although \cGrad{} is older than you, you can appreciate that \cGrad{} has a tougher home life than you with parents always fighting, sometimes literately. \cYoungest{}, your youngish \cYoungest{\uncle} will also be there. \cYoungest{They} are a hero in \pCityYoungest{}, you did not seem \cYoungest{\them} much, but \cTween{} has spend sometime out there. 


 
\begin{itemz}[Goals]
	\item Find enough money or valuealbe at \cGrandma{}'s house to rebuild your freeze and heat ray. You cannot simply as Grandma, but she would whole heartily approve of you steal it.  You can 
	%not sure how to write this because not sure how tinkering works yet
	
		\item Win prizes from \cGrandma{}'s scavenger hunt and beat your cousin and sibling
		\item Get one of your relatives to see you as an adult and let you city-sit for them next break
		\item Find out the whole story about the Chicago Incident and your \cAS{\parent}'s death
		\item Renew your relationship \cGrad{}. You used to be good friends, but have grown distance since \cGrad{} went to college and you have grown up a lot.
\end{itemz}

\begin{itemz}[Notes]
	\item 
\end{itemz}
\begin{contacts}
	\contact{\cGrandma{}} This is your Grandma and is a renowned villain, art thief, and occasional high profile celebrity kidnapper. Her house has lots of cool hidden parts and traps!
	\contact{\cOldest{}} This is your  \cOldest{\uncle} and is a villain. \cOldest{they} gave you the material and idea to build the freeze ray.
	\contact{\cOS{}} This is \cOldest{}'s spouse and you have a nice relationship with your \cOS{\uncle}.
	\contact{\cGrad{}} \cGrad{} is \cOldest{}'s child and your only cousin who just graduated from college. \cGrad{} was nice to you when you were younger, but has been gone for 4 years. The dynamic of your relationship needs to change. 
	\contact{\cYoungest{}} \cYoungest{} is your uncle and \cYoungest{\hero}. \cYoungest{} lives in \pCityYoungest{} and would be a great place to city-sit.  
	\contact{\cYS{}} \cYS{} is a hero and apparently been dating \cYoungest{} for a while. 
	\contact{\cArchitect{}} Your \cArchitect{\parent} who works as an architect in a powerful firm in \pCityArchitect{}.  You love your \cArchitect{\parent}, but feel like \cArchitect{\they} does not get you and wishes you to be more practical. 
	 	\contact{\cTween{}} Your younger \cTween{\sibling} who loves to find out what others would rather keep secret and loves a good conspiracy. \cTween{} is in 8th grade and is about to choose a high school. The decision must be made by tomorrow. 
\end{contacts}

\end{document}
