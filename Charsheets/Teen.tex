\documentclass[char]{LRSguildcamp1}
\begin{document}
\name{\cTeen{}}

\bigquote{``We don't want tradition. We want to live in the present and the only history that is worth a tinker's dam is the history we make today.''}{- Henry Ford}

\bigquote{``To invent you need a good imagination and a pile of junk.''}{- Thomas Edison}

Your name is \cTeen{\intro} and you are 16 years old. The best part about your life is your power. You have an innate knowledge of how to build gadgets and super-tech. Your skill grows with every passing project and some of your latest triumphs have included ice and heat rays. Your \cOldest{\uncle}, \cOldest{}, helped provide the necessary starting material for your ice ray and it was not too hard to create heat ray afterward. 

You attend \pNormalSchool{}, a non-super high school, in \pCityArchitect{}, which actually is ok with you. With your kind of power of heightened skill, you had a choice about whether to go to \pSuperSchool{} or not. At \pNormalSchool{}, you have friends and enjoy being the center of attention. You \textsl{could} be the top of your classes if you choose, but really don't see the point. As long as,  class work does not get in the way of your actual work, building cool stuff, its all the same to you. 

Another reason you like your school is that you are near the top of the social hierarchy, which admittedly is nice. The other students are always coming to you to fix their phone or add new feature to their car. With enough time and resources, you can build anything. You became a local legend a two weeks ago by freezing an annoying substitute teacher who had gone on a small tirade during class. It was worth the current suspension and the lecture by your \cArchitect{\parent}, \cArchitect{}. The substitute teacher deserved it. Though, your heat and freeze rays have been taken though and that sucks. 

While you are ok not attending \pSuperSchool{}, you still do want to live as a Super. You were afraid your talent would not be recognized at \pSuperSchool{} and you would have be slotted into a track B or C. You have been biding your time and honing your skills so you can debut as a super while in college. In private, you have sketched out what your costume and theme music might be. Its good to be prepared. \textit{You now just need a Super name.} You have even dressed up in nondescript clothes to try and pull some jobs. They went mostly fine -- your gadgets only sometimes blow up with minor explosions. It is all in the service of learning. If you don't push yourself into ever more extreme and varied situations, how can you be prepared? You are angling to get some more road tests for inventions next break. Your aunts and uncles all hold cities, if only you could convince one of them to let you city-sit for a while. Having access to city would be a great laboratory to try out new gear and prefect old ones. You just need to convince one you are mature enough.

Your \cArchitect{\parent}, \cArchitect{}, is alright but is mostly at the office all day. When you were younger, the two of you got along well. You would go to your \cArchitect{\parent}'s work and spend hours looking at plans for buildings. When your interests turned towards tech, something changed. Your \cArchitect{\parent} provides for you in the terms of work space and hack-a-thon style camps, but keeps not-so-subtlety trying to steer your interests towards the mundane world building and tech firms. 

Your younger \cTween{\sibling}, \cTween{}, is in 8th grade and undeniably a tween. \cTween{}'s a cool kid with a taste for ferreting out secrets, but has a bit of a wild imagination when it comes to conspiracy theories. \cTween{\They} is often wrong, but sometimes is surprisingly perceptive. \cTween{} is going to high school soon and will either go to your school, \pNormalSchool{}, or \pSuperSchool{}. You are not sure if you want to share your carefully crafted territory with \cTween{\them} or not. It could be useful to have a potential ally good at getting information. One major sticking point, though, is when you were at the same school before, \cTween{} would embarrass you by spouting ridiculous rumors loudly in the playground. You lost social standing. \cTween{} is older now and this might not happen again, but you can't be sure. \cTween{} must make a choice tonight to turn it in by tomorrow.

In school, you have recently read about the Chicago Incident in history class. While most of the world knows this is where some super hero trainees got carried away while heroing and damaged the city. For you, this is the incident where your \cAS{\parent},\cAS{} died. The details surrounding this are vague and you no longer feel satisfied with the story you were told as a child. That it was just a terrible coincidence. You sometimes feel distant from your\cArchitect{\parent} and \cTween{} about your \cAS{\parent}. Your \cArchitect{\parent} does not like to talk about \cAS{} and \cTween{} was only one and does not remember. You remember your parents were having problems right before your \cAS{\parent} died. Your \cAS{\parent} had taken you and \cTween{} from \pCityArchitect{} to stay with \cOS{} and \cGrad{} in \pCityO{}. Your \cArchitect{\parent} had come to the house where you were staying with \cGrad{} and \cGrad{\their} nanny. Your parents had left to go have an adult conversation, but your \cAS{\parent} never came back. You and \cTween{} went back to your \pCityArchitect{} and there was a funeral a few weeks later. It felt like your world had been torn apart and the wound never really healed. You mostly have tried not to think about it, but talking about the Chicago Incident has brought everything to the forefront of your mind.

Grandma has called a family meeting in \pCityGrandma{}. Your extended family will be at there as well. You have decided to use this chance to learn the truth about your \cAS{\parent} from the rest of your family. Also she contacted you to inform you that \cGrandma{\they} will be holding a secret competition for you, your \cTween{\sibling}, and your cousin with several prizes. \cGrandma{} has never been one to jest about cool gifts. You are intrigued.  

\cOldest{}, your \cOldest{\uncle} who is a villain, and his lover, \cOS{} who is a hero, will be there. You like \cOS{} better than \cOldest{}. \cOS{\They} is nicer than \cOldest{} and controls electricity. You two have collaborated on some tinkering projects over the summers. Regardless, it has been good for your rep to be related to this super couple odd pairing. You cannot say their relationship is loving but has been going on for a long time. No one is allowed to mess with their city but them.  You would often spend summer vacations with their child, \cGrad{}, before \cGrad{\they} went off to college. Although \cGrad{} is older than you, you can appreciate that \cGrad{} has a tougher home life than you with parents always fighting, sometimes literately. You used to be on of the only people who could get \cGrad{} to smile.

\cYoungest{}, your younger \cYoungest{\uncle}, will also be there. \cYoungest{They} is a popular hero in \pCityYoungest{}. You have not seen \cYoungest{\them} much, but \cTween{} has spend sometime out there. You have heard that \cYoungest{} is dating \cYS{}, another hero in \pCityYoungest{}.


 
\begin{itemz}[Goals]
	\item Find parts at \cGrandma{}'s house to rebuild your freeze and heat ray. You cannot simply ask Grandma for them, but she would whole heartily approve of you stealing the parts. See Tinkering. 
	%not sure how to write this because not sure how tinkering works yet
	\item Win prizes from \cGrandma{}'s scavenger hunt and beat your cousin and sibling
	
		\item Get one of your relatives to see you as an adult and let you city-sit for them next break
		
		\item Choose a Super name for yourself
		
		\item Decide whether or not to encourage \cTween{} to come to your high school.
		
		\item Find out the whole story about the Chicago Incident and your \cAS{\parent}'s death
		
		\item Renew your relationship \cGrad{}. You used to be good friends, but have grown distance since \cGrad{} went to college.
		
\end{itemz}

\begin{itemz}[Notes]
	\item 
\end{itemz}
\begin{contacts}
	\contact{\cGrandma{}} This is your Grandma and is a renowned villain, art thief, and occasional high profile celebrity kidnapper. Her house has lots of cool hidden parts and traps! Grandma lives in \pCityGrandma{}, which would be an awesome place to city-sit. 
	
	\contact{\cOldest{}} This is your  \cOldest{\uncle} and is a villain. \cOldest{they} gave you the material and idea to build the freeze ray. \cOldest{} lives in \pCityO{} and would be an another good place to city-sit.
	
	\contact{\cOS{}} This is \cOldest{}'s spouse and you have a nice relationship with your \cOS{\them}.\cOS{} lives in \pCityO{} and would be an another good place to city-sit.
	
	\contact{\cGrad{}} Your only cousin  and has just graduated from college. \cGrad{} had always been nice to you when you were younger, but has grown distant over the last 4 years.  
	
	\contact{\cYoungest{}} \cYoungest{} is your \cYoungest{\uncle} and \cYoungest{\hero}. \cTween{} has spent more time with your \cYoungest{\uncle}. \cYoungest{} lives in \pCityYoungest{} which would be a great place to city-sit though.
 
	\contact{\cYS{}} \cYS{} is a hero and apparently been dating \cYoungest{} for a while. 
	
	\contact{\cArchitect{}} Your \cArchitect{\parent} works as an architect in a powerful firm in \pCityArchitect{}.  While you love your \cArchitect{\parent}, you feel a bit distant from \cArchitect{\them}. Currently, you are annoyed that your 
	
	\cArchitect{\parent} took away your ice and heat ray after the incident with the substitute teacher. 
	
	 	\contact{\cTween{}} Your younger \cTween{\sibling} who loves to find out what others would rather keep secret. \cTween{} is in 8th grade and is about to choose a high school. The decision must be made by tomorrow. 
		
\end{contacts}

\end{document}
