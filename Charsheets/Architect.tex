\documentclass[char]{LRSguildcamp1}
\begin{document}
\name{\cArchitect{}}

\bigquote{``Architecture should speak of its time and place, but yearn for timelessness.''}{- Frank Gehry}

Your name is \cArchitect{\intro} and you are an Architect. You are 40 years old and your professional life is near prefect. You are the head Architect at a firm headquartered in \pCityArchitect{} and entertain offers from across the world.  You are secure enough that you can pick and choose your clients: As Ayn Rand observed, 'I don't build in order to have clients. I have clients in order to build. Your designs balance the functional needs with a sense of elegance and purpose. From childhood, you have always found a sense of beauty and satisfaction in the clean, ordered lines of a blueprint and the mathematics that underpins the system. Your skill at your chosen craft has always received praised and admiration from others, except from the one place you secretly want it the most--your family. Recently, you have a renewed interest of building something that even will command respect from them and the prefect chance just came along. The local hero and villain teams have just put in bids for your company, now which one to choose.

Your family is super in the most literal sense. Your \cGrandma{\parent}, \cGrandma{}, is a renowned \cGrandma{\villain}, art thief, and occasional high profile celebrity kidnaper. \cGrandma{\They} have held the city of \pCityGrandma{} for years. Your two sibling have always cast a shadow over your life. Being the middle child supposed to be challenging but your situation borders on ridiculous. Your older \cOldest{\sibling}, \cOldest{} has always followed \cGrandma{} down the path of villainy and was the golden child through your younger years. Just when \cOldest{} moves out, your younger \cYoungest{\sibling}, \cYoungest{} is born and takes the lion share of attention. \cYoungest{\They} as the baby of the family always get off scotch-free, even becoming a \cYoungest{\hero} with \cGrandma{} brushes it off as nothing. Before \cYoungest{} came along though, the one shining part of your home life was your \cGS{\parent} \cGS{}. \cGS{} understood always supported your interests and passions, although admittedly not sharing them. 

During take your child to work days, \cGS{} would plan \cGS{\their} heists around the latest building that had caught your interests and take you to the most restricted places. \cGS{} would laugh over \cGS{\their} shoulder saying that nothing was off limits for \cGS{\their} kid. Your favorite adventure was when \cGS{\they} took you to eat donuts on top of the Statute of Liberty and remarked how men can live forever.  '' You know, \cArchitect{}, some \cGS{\humans} like me pull these big jobs so we will be remember, but looking at you I see another way. You are going to build something magnificent something people will remember and use for generations. I'm so proud of you.'' Your \cGS{\parent} died a month later of a heart attack not long after \cYoungest{} was born. That day is emblazoned into your memory and is the reason you chose to base your firm in \pCityArchitect{}.

You yourself do have powers but where other people seem to orbit their life around them -- you view yours merely as a useful tool. Your power is nullification, which gives you the ability cancel out other powers. This comes in use when sometimes taking contract with super clients. Your power is not who you are. \cGrandma{} has told you countless times that you are wasting your true talent. Despite what \cGrandma{} 's contends, you have a deep-seated suspicion that \cGrandma{\theyare} afraid of your power, because it makes \cGrandma{\them} feel weak. You have not received a hug from your \cGrandma{\parent} since your power manifested. Its something you don't like to examine too closely but since having children of your own \cTeen{}, your oldest, and \cTween{}, your youngest, you have tried to demonstrate your love by providing them a stable life. This has met with mixed success, but overall you have a loving relationship with your children.  There is one secret, however, you have been keeping from them. 

You were married to \cAS{}, who you met a at college while studying architecture. \cAS{} was a breathe of fresh air, while \cAS{\they} had powers, they still choose to purse a more typical degree and seemed wonderfully down to Earth. \cAS{} even managed to get along with your older \cOldest{\sibling}, \cOldest{} and \cOldest{\their} \cOS{\spouse} \cOS{}. You got married and had your two kids. You thought everything was going alright and you were working hard to build a stable life for your new family.  Thirteen years ago though, \cAS{} started to see things differently and \cAS{\they} explained how they felt confined and ignored, so requested a divorce. \cAS{} took your children, \cTeen{} who was 6 and \cTween{} who was only 1. They moved into \cOldest{}'s \pCityO{}. You went to confront \cAS{} for taking the children suddenly and without any formal agreement.  While in the midst of the argument, the walls in the room began to shake and the western wall caved as if it were made of paper of a young \cJuggernaut{\hero}-in-training \cJuggernaut{} came barreling through the wall. \cAS{} ended up caught off guard and was trampled as \cJuggernaut{} could not stop the momentum. You were in shock and unable to react. You never had the special training those who go to the super high school received. There was nothing you could do, except for a small voice in your head that whispers your powers could have stopped \cJuggernaut{}. \cJuggernaut{} continue \cJuggernaut{\their} path of destruction leaving you with the tableau of mangled and disfigured body of the person who was once the love of your life. You ran. After several hours, you returned to the scene to find \cOS{} and \cOldest{} along with the police coordinating off the scene. You learn that no one had seen you and you did not correct them. Your children were returned to you that very day and you began funeral arrangements. You are not proud of your behavior either your freezing or silence, but what good would it do to tell. Over the next 13 years, you have build a secure life style for children and trying to mostly ignore the drama of your super family. 

\cGrandma{} has called the family together for an announcement. \cGrandma{\They} told you mostly \cGrandma{\they} wanted to see \cGrandma{\their} grandchildren and bemoaned how you never visit.  You suspect that is announcement might be about \cGrandma{\their} retirement as the retirement age is 65 for most heroes and villains and \cGrandma{\theyare} already 70.  


\begin{itemz}[Goals]
	\item Decide which competing bid for a building an HQ for either the Hero Team or the Villain Team
	\item Convince \cTween{} to go a non-super high school like you did.  
	\item Convince \cTeen{} to use \cTeen{\their} talent for design to move into a STEM field and not supering.
	\item Keep a positive relationship with your children
	\item Improve your relationship with your mother and receive a willing hug from your mother.  
	\end{itemz}

\begin{itemz}[Notes]
	\item 
\end{itemz}



\begin{contacts}
	\contact{\cGrandma{}} This is your \cGrandma{\parent} and is a renowned villain, art thief, and occasional high profile celebrity kidnapper. \cGrandma{\Theyare} older and already past the age of retirement for most villains.
	\contact{\cOldest{}} This is your older \cOldest{\sibling} and you lived in \cOldest{\their} shadow growing up. \cOldest{\They} were good friends with your spouse before \cAS{\their} death. \cOldest{} is a villain. 
	\contact{\cOS{}} This is \cOldest{}'s spouse. You have a friendly relationship with \cOS{\them}. As a hero, you truthfully you don't know how their relationship works, but they seem happy if dysfunctional.  \cOldest{} and \cOS{} are constantly fighting over who is in charge of their city.  Over the last 20 years, the city has changed hands so many times you have lost count and decided that it does not really matter who it is. 
	\contact{\cGrad{}} \cGrad{} is \cOldest{} and \cOS{}'s only child and just graduated from college. 
	\contact{\cYoungest{}} \cYoungest{} is your younger \cYoungest{\sibling} and a \cYoungest{\hero}. You think \cYoungest{they} have always been your \cGrandma{\parent}'s favorite who can do no wrong. 
	\contact{\cYS{}} \cYS{} is a \cYS{\hero} and apparently been dating \cYoungest{} for a while. You have not met them before. 
	\contact{\cTeen{}} Your oldest child who is a high school student at the \pNormalSchool{}. \cTeen{\They} have a flair for building super tech, including heat and freeze rays. In fact, \cTeen{} got in trouble for freezing a substitute teacher and subsequently lost the rights to said heat and freeze ray two weeks ago. \cTeen{} is currently on suspension from school.
	\contact{\cTween{}} Your youngest child is about to decide what high school to enter. \cTween{} loves a good mystery and is always trying to sell his latest theories. This has led to some trouble at school last year in 7th grade, for example when \cTween{\they} wrongly accused two sets of teachers making out in the science room. That being said, \cTween{\they} was right the third time.
\end{contacts}

\end{document}
