\documentclass[char]{LRSguildcamp1}
\begin{document}
\name{\cOS{}}
You are \cOS{\intro}, the nationally-known \cOS{\hero} of \pCityO{}, defender of its citizenry and upholder of its laws...sort of.

You first set foot in \pCityO{} twenty-two years ago, a young \cOS{\hero} trying to find \cOS{\themself}, and find a cause.  You quickly became aware of the presence of a shadowy villain, \cOldest{\MYsupername} (real name \cOldest{\intro}), and caught them by surprise.  % I am tempted to have them each remember the other getting the upper hand...
You found yourself locked in combat with the \cOldest{\villain}, brawling through the streets of \pCityO{}, wielding your power over electricity against \cOldest{\their} powers over ice.  You battled for hours, until you finally wrestled \cOldest{\them} to the ground -- and then you found yourselves locked into a passionate embrace.  What can you say?  For a villain, \cOldest{} is dashing, audacious, and...quite attractive, honestly.

A strange sort of relationship blossomed.  You have always been on opposite sides of the battle for good and evil, and you are constantly striving with \cOS{\them} for control of the streets of \pCityO{}, but you can't bring yourself to do \cOS{\them} any lasting harm, and the times you've managed to tie \cOS{\them} up -- or \cOS{\they} have tied you up -- have ended in a bedroom, not a cell.

Nine months after that first encounter, you had a \cGrad{\offspring}, \cGrad{\intro}.  
You and \cOldest{} negotiated a part-time truce -- a few days a week -- to try to have some sort of relationship, and raise your \cGrad{\offspring}.  
The effectiveness of that arrangement has waxed and waned over the years, largely due to the amount of time you spend fighting, and the amount of time you spend in the bedroom.
Or on the roofs of buildings.  
It's that sort of a relationship.  

You found it hard to relate to \cGrad{}, as your relationship with \cOldest{} and your obligations to your city kept you busy.
It's not all your fault -- \cGrad{\their} power of invisibility makes \cGrad{\them} easy to miss even when you are looking for \cGrad{\them} -- but you've never felt like you were able to connect to \cGrad{\them}.  
By the time \cGrad{\they} had grown into a young adult, you felt like \cGrad{} had drifted away from you, and it didn't help when \cGrad{\they} moved away to boarding school and then college.
\cGrad{\They} graduated a few months ago, and moved back home, but has been reticent about what \cGrad{\they} learned, or what \cGrad{\they} want to do with \cGrad{\their} lives.  \cGrad{\They} haven't yet indicated an interest in striking out as a hero or a villain.  
You have occasionally heard strange humming sounds coming from \cGrad{\their} room, and they seem to be on the phone a lot, but you haven't worked out what exactly is going on there. % sewing machine ;)
Nevertheless, you want to use this family occasion as a chance to bond again.

Recently, \cGrandma{\MYsupername} (real name \cGrandma{\intro}), \cGrandma{\villain} and \cGrandma{\familyleader} of the \cGrandma{\formal} family, asked \cGrandma{\their} entire family to join \cGrandma{\them} at \cGrandma{\their} lair in \pCityGrandma{}.  You received an invitation, too -- probably because \cGrandma{\they} knew you wouldn't want \cGrad{} to go without you, and because they all consider you a sort of family.  You decided to come along, hoping to find time to bond with \cGrad{}, and to play role model for the younger generation.  You'd hate to see the \emph{whole} family go villainous if you can help it.

You've actually done a decent amount of bonding with the extended family over the years you've been with \cOldest{}.  You spent a lot of time with \cOldest{\their} younger \cArchitect{\sibling} \cArchitect{\intro}, their \cAS{\spouse} \cAS{\intro}, and their children \cTeen{} and \cTween{}. You bonded pretty well with the six-year-old \cTeen{}, the older of the two, who you quickly found common ground with -- their technological wizardry shared interesting details with your powers over electricity.  You, the young \cTeen{}, and \cAS{} all got along quite well.

When \cTeen{} and \cTween{} were relatively young, however, there was some drama between \cArchitect{} and \cAS{}, and \cAS{} and the kids came out to live with you in \pCityO{}.  At the time, you were pleased with your role in the \cHeroLeague{\intro}, and had volunteered to take two young heroes under your wing, \cJuggernaut{}, a fast, strong, and muscular superhuman with family connections very high in the \cHeroLeague{}, and \cYS{\MYsupername}, who had the power to supercharge other superhumans.  While you were out in the streets doing heroism, however, something went horribly wrong.  You returned later that night to discover \cAS{} dead in a pile of rubble, and \cJuggernaut{} telling the police that \cYS{\MYsupername} had irresponsibly overcharged \cJuggernaut{}'s powers, and an accident had happened.  The story that made it to the media called it the \pCityO{} Incident, and placed the blame firmly on \cYS{\MYsupername}'s shoulders, but you don't think that's the whole story.  \cYS{} was quickly ushered out of the \cHeroLeague{} in disgrace, before you had the chance to get \cYS{\their} side of the story.  The kids, too young to remember much, were placed back in \cArchitect{}'s custody, and know very little of what happened; you haven't dared to bring it up with them when you've seen them.

As a result of the whole incident, you've grown distrustful of the \cHeroLeague{\intro} and its policy of assigning new and inexperienced heroes to new territories.  You have known \pCityO{} since your first days of heroing, and that's the way it should be: the heroes who know a city should stay there, and work without the interference of others -- look how much damage two newbie heroes caused.  If you could leave the \cHeroLeague{}, but continue to protect the citizens of \pCityO{} as a solo hero, you'd do it in a heartbeat -- but you still have something of a shadow hanging over you from the \pCityO{} Incident, and if you left while the \cHeroLeague{} was in significant public favor, they'd probably send enough force to \pCityO{} to take it from you and \cOldest{} entirely.

% TODO: add a mempacket that affects OS' influence score when they leave the League

That said, you and \cOldest{} \emph{are} constantly in a subtle war for control over \pCityO{}, trying to gain the upper hand over each other.  Right now \cOldest{} has a bit of an edge, but you think with the right word to the right people in the police department back home, you might be able to have an edge when you get back to \pCityO{}.  In the meantime, you know \cOldest{} is trying to improve \cOldest{\their} standing with the Villain Compact, but you {\em like} the balance you have in \pCityO{}.  If they got big promotions in the Compact, they might gain the strength to take \pCityO{} permanently, or want to start expanding beyond the city, or just stop paying as much attention to you.  You don't want to {\em destroy} their career, just...keep it from going anywhere new.

\begin{itemz}[Goals]
	\item Do not allow anyone other than you or \cOS{} to interfere with \pCityO{}.  % That means Grandma, mostly.
	\item Have your allies attack and take \pCityO{} away from \cOldest{}.
	\item Embarrass and escape the \cHeroLeague{\intro}.
	\item Embarrass \cOldest{} in front of the \cVillainCompact{\intro}.
	\item Reestablish a relationship with \cGrad{}.
	\item Convince as many youngsters as you can to become heroes...but you'd prefer them to be independent heroes, not part of the \cHeroLeague{}.
\end{itemz}

\begin{itemz}[Notes]
	\item While you find \cOS{} extremely hot, you're among family.  Displays of aggressive affection will not be appropriate.  (Translated: don't make the other players uncomfortable.)
\end{itemz}

\begin{contacts}
	\contact{\cOldest{}}  The \cOldest{\villain} you've been in a torrid romance with for twenty-two years.
	\contact{\cGrad{}} Your \cGrad{\offspring}, who you feel you've grown distant from.
	\contact{\cGrandma{}} \cOldest{}'s \cGrandma{\parent}, a legendary thief and supervillain.
	\contact{\cArchitect{}} \cOldest{}'s middle sibling, who went into architecture instead of heroism or villainy. 
	\contact{\cTeen{}} Your older \cTeen{\nephew}, a technical wizard who is a bit of a hellion.  You like \cTeen{\them}, though, as their powers and yours have some interesting commonalities.
	\contact{\cTween{}} Your younger \cTween{\nephew}, a charmingly naive conspiracy theorist.  Entering high school shortly.  They grow up so fast.
\end{contacts}

\end{document}
