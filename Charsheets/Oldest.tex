\documentclass[char]{LRSguildcamp1}
\begin{document}
\name{\cOldest{}}

You are \cOldest{\intro}, the eldest child of \cGrandma{\intro}, and a \cOldest{\offspring} fit to carry on the family name of villainy.  
From childhood, you learned to wield the powers of ice and cold to inspire fear in your enemies, subtle or overt: an innocent-looking patch of black ice here, an icicle to the throat there...all very much the same. 
\updatemacro{\cGrandma}{
  \changename{Mother}
}
Just like \cGrandma{} would do.  That's why you're \cGrandma{\their} favorite: you always wanted to follow in \cGrandma{\their} footsteps, always modeled yourself after \cGrandma{\them}. % totally false :P

You have two younger siblings, \cArchitect{\intro} and \cYoungest{\intro}.  \cArchitect{} has the power to temporarily disable other superpowers on touch -- whether \cArchitect{\they} meant to or not -- and that tends to make you, \cYoungest{}, and \cGrandma{} nervous.  Your \cGS{\parent} \cGS{\intro} was reasonably fond of you, but clearly favored \cArchitect{}.  However, \cGS{} died of natural causes shortly after the birth of \cYoungest{}. \cYoungest{} was always \cGrandma{}'s little baby, and got away with just about everything -- up to and including becoming a \cYoungest{\hero}.  You were already sixteen when \cYoungest{} was born and your \cGS{\parent} died, however, already primed to become \cGrandma{}'s true heir of villainy.  \cArchitect{} went on to become, of all things, an architect -- \cArchitect{\they} didn't seem to care to use \cArchitect{\their} powers at all.

You graduated from super college, and began to roam, looking for a city to call your own.  Shortly after turning twenty-three, you encountered a wandering superhuman in the streets of \pCityO{}.  This, you learned, was \cOS{\intro}, a young \cOS{\hero} with powers over electricity, eager to find villains to fight and civilians to protect.  \cOS{\They} got the drop on you, but you were a match for \cOS{\them}.   Through the streets and skyscrapers, you found yourself locked in combat; you battled for hours, until \cOS{\they} finally wrestled you to the ground -- and then you found yourselves locked into a passionate embrace.  What can you say?  \cOS{} is, honestly, extraordinarily hot for a hero.

A strange sort of relationship blossomed.  You have always been on opposite moral sides -- your amoral ambition, and their pragmatic determination to protect the populace -- and you are constantly striving with \cOS{\them} for control of the streets of \pCityO{}, but you can't bring yourself to do \cOS{\them} any lasting harm, and the times you've managed to tie \cOS{\them} up -- or \cOS{\they} have tied you up -- have ended in a bedroom, not a cell.
Even though you fight with \cOS{} over \pCityO{}, you're in perfect agreement that if anyone controls \pCityO{}, it is going to be one of you two.  Villains and heroes alike who have tried to muscle in on \pCityO{} have found themselves simultaneously beset by bolts of electricity and hails of icicles.

Citizens of \pCityO{} have learned to accept the situation. While they often breathe a sigh of relief on the days you two observe a truce, many now find your fiery relationship entertaining, as your fighting has relatively little impact on their day to day lives.  (You have heard rumors of fanfiction, though you have so far resisted the urge to see for yourself.)

Nine months after that first encounter, you had a \cGrad{\offspring}, \cGrad{\intro}.  
You and \cOS{} negotiated a part-time truce -- a few days a week -- to try to have some sort of relationship, and raise your \cGrad{\offspring}.  
The effectiveness of that arrangement has waxed and waned over the years, largely due to the amount of time you spend fighting, and the amount of time you spend in the bedroom.  
Or on the roofs of buildings.  
It's that sort of a relationship.  

You like your \cGrad{\offspring} well enough, but you have to admit you overlooked \cGrad{\them} a lot in their childhood.  
It's not all your fault -- \cGrad{\their} power of invisibility makes \cGrad{\them} easy to miss even when you are looking for \cGrad{\them} -- but you've never really managed to relate to \cGrad{\them}.  
By the time \cGrad{\they} had grown into a young adult, you had practically given up on learning to relate to \cGrad{\them}; you almost didn't notice when \cGrad{} went off to college.  
\cGrad{\They} graduated a few months ago, and moved back home, but has been reticent about what \cGrad{\they} learned, or what \cGrad{\they} want to do with \cGrad{\their} lives.  \cGrad{\They} haven't yet indicated an interest in striking out as a hero or a villain.  
You have occasionally heard strange humming sounds coming from \cGrad{\their} room, but you haven't worked out what exactly is going on there. % sewing machine ;)

Despite your incomplete control of \pCityO{}, you have, through steady skimming of protection fees and other resources, gained significant rank in the Villain Compact.  You've recently been offered a promotion to Secretary, which would give you more resources and bring you to more national prominence -- but first you must prove yourself.  The Compact has bid for your \cArchitect{\sibling} \cArchitect{}'s assistance in building a new national villain headquarters, but \cArchitect{} has not yet accepted the offer -- and as difficult as your relationship is, you've still got the best chance of any villain to convince \cArchitect{\them}.  \cArchitect{} may be too difficult to threaten directly, with their ability to nullify your powers, but you think you might be able to add some spice to the deal, perhaps by offering some interesting items, cash, or, perhaps, doing something nice for one of \cArchitect{}'s kids.  

They're pretty nice kids, actually, and they've shown interesting hints of potential villainy.  Perhaps you could recruit them into the Junior Villains League?  That, too, might impress the Compact.  The older one, \cTeen{} has technical wizardry with all sorts of applications; the Compact would very much like to purchase some of \cTeen{\their} designs.  The younger, \cTween{}, has super hearing -- and while that's not necessarily a very exciting power, it'd come in awfully handy in knowing just which political figure to blackmail.

\cGrandma{} has invited you to a family gathering.  You've suggested retirement once or twice, as \cGrandma{\theyare} getting on in age, and that could present a major opportunity for you to inherit \cGrandma{\their} minions, resources, and technology.  \cGrandma{\They}'d surely want to pass them to \cGrandma{\their} eldest \cOldest{\offspring}, the \cOldest{\villain} who followed in \cGrandma{\their} footsteps,  right?

\begin{itemz}[Goals]
	\item Do not allow anyone other than you or \cOS{} to interfere with \pCityO{}.  % That means Grandma, mostly.
	\item Coordinating with your minions in \pCityO{}, avoid letting \cOS{} manage to take control of the city while you're gone.
	\item Make sure you will inherit \cGrandma{}'s supervillain resources. % Red herring, not actually possible.  Check if that's okay in the app.
	\item Improve your application to become Secretary of the Villain Compact through as many of these objectives as you can complete:
	\begin{itemize}
		\item Convince \cArchitect{} to accept the Villain Compact's bid for \cArchitect{\their} services in constructing a new HQ.
		\item Several of your unaligned family members might be interested in a career in villainy.  Convince as many as possible to join the Villain Compact, or for the younger set, the Junior Villains League.
		\item Take any opportunity to disgrace the League of Heroes.
		\item Take any opportunity to defend the public image of the Villain Compact.  % Not actually going to be a thing.
	\end{itemize}
\end{itemz}

\begin{itemz}[Notes]
	\item You do like your family...even the heroes, you suppose.  You're not likely to make personal threats against them or their loved ones, especially not the younger generation.  On the other hand, if you find one of your generation has left death rays or other valuables where you or your minions can get at them...well, you \emph{are} a \cOldest{\villain}.  And when it comes to \cGrandma{\Parent}, you know \cGrandma{\they} will be proud of you if you can steal anything from \cGrandma{\them}.
	\item While you find \cOS{} extremely hot, you're among family.  Displays of aggressive affection will not be appropriate.  (Translated: don't make the other players uncomfortable.)
\end{itemz}

\begin{contacts}
	\contact{\cOS{}} The \cOS{\hero} you've been in a torrid romance with for twenty-two years.
	\contact{\cGrandma{}} Your \cGrandma{\parent}, \cGrandma{\villain} extraordinaire.  You are \cGrandma{\their} favorite child.
	\contact{\cArchitect{}} Your middle sibling.  Makes most of the family, you included, uncomfortable to be around, but you need them to accept the Villain Compact's bid for a new HQ.
	\contact{\cYoungest{}} Your youngest sibling.  \cGrandma{\Parent} let \cYoungest{\them} get away with everything, up to and including becoming a \cYoungest{\hero}.  A little snot.
	\contact{\cGrad{}} Your wayward \cGrad{\offspring}, who you feel like you've lost touch with.
	\contact{\cTeen{}} \cArchitect{}'s older child, a technical wizard.
	\contact{\cTween{}} \cArchitect{}'s younger child, has super-hearing.  Runs a conspiracy theory blog.  
\end{contacts}

\end{document}
