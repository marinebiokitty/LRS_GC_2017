\documentclass[char]{LRSguildcamp1}
\begin{document}
\name{\cGrandma{}}

You are the supervillain \cGrandma{\MYsupername} (real name \cGrandma{\intro}) and one of the last solo villains holding your ground. The very first thing you stole was a puzzle box. The box was small a gilded enameled Victorian sort of thing that was valuable enough to be locked up and displayed in its glass case. You were 5. 

Thus began your illustrious career in theft. At first, it was small things that no one noticed. How could you resist? After all, you could simply just reach through all the barriers and have anything you wanted. Your tastes grew in sophistication, and you amassed a small museum filled with robust Rubenesque pieces in vivid colors. The more complicated the heist, the more of a thrill it gave you. 

Your parents sent you to \pSuperSchool{} hoping that the school would be able to control you. Being the Queen Bee came as  easily as stealing to you, but you quickly grew bored with the simpering flocks of geese jockeying to curry your favor. \pSuperSchool{} cramped your style and in your heart you knew you belonged somewhere else. You wonder what would have happened if you'd gone to art school, where you could be around beautiful things and break the rules all the time. 

After a while, stealing with your powers became too easy. You became obsessed with escapology and Houdini and the normal ways of picking a lock or sleight of hand. At 17 years old, you ran away from home to \pCityGrandma{} and joined a magicians act. At first you were only a lowly assistant, but your fearlessness and flair for the dramatic eventually led to you taking over the show. In those years, you acquired a keen understanding of how to perform and how to give people what they want to see. 

You still continued to visit the big museums in the city, and by visit you mean rob in a relatively low-key manner. You would sometimes steal art supplies and leave them as an anonymous gift for artist you thought seemed promising. You do understand the importance of giving back. 

One night, during a show you looked out in the crowd and locked eyes with a man, \cGS{\intro}. You would find out late that he was known to public as \cGS{}, the super villain terrorizing the extended area of \pCityO{}. Even though you swiped \cGS{\their} wallet that night, he took your heart. 

You thought that \cGS{\intro} understood you in a way that no one had before. For your honeymoon, \cGS{\they} took you to Vegas and together you looted the casinos blind. \cGS{\They} built you your first super-lair with no doors, because you could phase through walls anyways, that you filled with the things that you stole together. For a while you were happy. 

It didn't last long. Being with him meant that you had to be part of the larger world of super-villainy, that Villain Compact group that you had tried so hard to leave behind in \pSuperSchool{}. Still, you made lemons out of lemonade. To this very day, the Villains still speak your name with reverence, especially about the Incident of '62. 

During this decade, you became pregnant with \cOldest{} and then \cArchitect{}. At first you thought that nothing in your life would have to change. When \cOldest{} was small enough to be carried around in a sling, you took \cOldest{\them} with you on heists. However, when you were pregnant with \cArchitect{} suddenly you powers fluctuated wildly leaving you helplessly stuck in a wall at times. You freaked out, thought you were losing your powers. 

The births of the children also bought more problems into your relationship, inciting quarrels between you and \cGS{\intro} that would last for days. \cOldest{} had so much promise, but in the end he followed \cGS{\intro} down the road of conventional villainy. As for your other child, \cArchitect{} was always strange and difficult to bond with. You still feel a bit eerie when you have to come into contact with \cArchitect{} , reminded of that time during your pregnancy that you were so helpless. \cGS{} coddled \cArchitect{\them}, which is probably why \cArchitect{\they} ended up pursuing a silly normal career. 
After you recovered,  it only took a year until you had securely gripped \pCityGrandma{} in your control. Your reputation was able to keep even the most enterprising young villains away. Sometimes the League would assign someone, who was amusing enough, to counter you. 

Your main sources of dissatisfaction were in your home life. You were also aware that \cGS{\intro} was jealous of your prestige. The attraction was still there, but it was more often fueled by resentment and trying to get a hand over each other.  You both tried to keep up a good front for the children, but things spilled over, especially as \cOldest{} left for College. You spent more and more of your own free time making mediocre art in your lair, dissatisfied that all your creations had come to naught. 

The birth of \cYoungest{} was an unexpected gift. And then \cGS{\intro} passed away in an embarrassingly normal heart attack after \cYoungest{} was born. And you felt... relief. Even though \cGS{\intro} was the father of your children and someone you once loved, your relationship had never really gone back to the way it used to be.  

You allowed \cYoungest{} free rein with their powers, pushed and nudged them to grow into the person they were meant to be instead of becoming a copy of yourself. Certainly, you weren't happy when \cYoungest{\they} decided to become a \cYoungest{\hero}. But at the same time you were proud. Unlike \cOldest{} who was still walking the path of mediocre villainy and carrying on an embarrassingly torrid relationship with that \cOS{\hero}. Something out of a daytime soap! 

Suddenly you have become old. Where did all those years go? Half a century of being a celebrated\cGrandma{\villain}, and decades of being an anonymous artist. Your artwork has gotten better and better over the years, so much that it's been collected by the MoMA, the Whitney, the Philadelphia Museum of Art. Some of it hangs in your house, perfectly blending in the whit the rest of the artwork you have acquired over the years. You wish there was someone to share this other side of you with. 

Recently you took a hit to the knee and had actually take a gun to threaten your foes. You have never had a taste for violence, it is far too vulgar. Then \cOldest{} even had the audacity to use the word retirement! You are just not being respected enough by the children these days,who think that their gadgets and fancy groups are the solution. When you carry out your masterplan to rob their cities at this dinner, then they'll finally remember who is the boss.  You have many minions on the move for while your kids are distracted.  They should arrive at your place at 3, and your minions will be ready for raids at the following times:

\begin{tabular}{|l|c|l|}
{\bf City} & Current defender & {\bf Time} \\
\pCityYoungest{} & \cYoungest{} & 3:30 \\
\pCityO{} & \cOldest{} & 3:45 \\
\pCityYoungest{} & \cYoungest{} & 4:15 \\
\pCityO{} & \cOldest{} & 4:30 \\
Your choice of \pCityYoungest{} or \pCityO{} & ? & 5:00
\end{tabular}

Another problem is that nowadays you often have to rely on minions, and even acceptable minions are hard to come by now. You haven't seen your other children and grandchildren in some time and wonder how they are. Are they still so easily manipulated? You desperately need a good side-kick who preferably can patiently deal with numbskulls.  At least you have a few talented Grandchildren to pick from. Finally, the engagement rings will be put to good use to communicate with your sidekick, whoever that turns out to be. 

To choose a sidekick, you have planned a beautiful challenge for your grandchildren. Something that plays to their talents, but is still easy enough for them to solve. You know the value of letting youngsters build up their confidence. They don't necessarily need to solve it, but just to demonstrate their powers. You know that \cTeen{} is eager to prove themselves and their tinkering capabilities. 

You've purposefully left out some gifts. Your other grandchildren, \cGrad{} and \cTween{}, have equal possibilities for villainous careers, once you get your pesky children out of the way! It is really a shame that \cYoungest {} never inherited your passion for art, but now is the time to find a grandchild that has? 

It's also time for \cYoungest {} to settle down. Although you are not sure you like the sound or the look of that young \cYS{\hero} that \cYoungest{} has been shacking up with. Perhaps you will meddle... a bit. 

Watched the Avengers for the eye candy and the costumes. Fetched \cChrisHemsworth{\intro} from his trailer and locked him up in your lair for enjoyment. He is far more enjoyable than that Downey Jr. fellow who talked your ear off. 

\begin{itemz}[Goals]
	\item Distract children throughout dinner with personally invasive questions and by pitting them against each other. 
	\item While children are distracted, run secret obstacle course for grandchildren to pick a new sidekick. 
	\item While everyone is distracted, rob children's cities blind. 
	\item Find out which one of your grandchildren will be your spiritual successor, artistically. 
	\item Announce that Grandma is back in business. 
	
\end{itemz}

\begin{itemz}[Notes]
	\item \cOldest{} and \cOS{} are in constant conflict for \pCityO{}.  If \cOldest{} thinks that your minions actually belong to \cOS{}, and you offer help, you may be able to learn \cOldest{\their} defense plans...
	\item You have freed up a fair amount of cash for tonight, which will come in handy during your raids on other cities, but other uses for money may appear this afternoon.
\end{itemz}

\begin{contacts}
	\contact{\cOldest{}} Your oldest \cOldest{\offspring}. Promising as a child, but ultimately disappointing. 
	\contact{\cOS{}}  Your oldest \cOldest{\offspring} spouse. Too opinionated and too heroic. Their torrid romance is quite overdone. 
	\contact{\cGrad{}} Your oldest grandchild. Shows some promise with their powers,but has never been too receptive to your mentoring. Seems like Grad might bring shame to the family name.  
	\contact{\cArchitect{}}  Your second child. you have an awkward relationship and nothing in common. Their personality also makes you slightly uncomfortable. 
	\contact{\cTeen{}} Your second oldest grandchild. Inherited your mind, but not your abilities. You have some warmth towards Teen and pity the child for being trapped with in a non-super household. This one lacks backbone. 
	\contact{\cTween{}} Your youngest grandchild. Your favourite grandchild, because this one has spunk and does not follow the rules. Useful abilities, but needs discipline and training. Too easily distracted. 
	\contact{\cYoungest{}} Your youngest child. Too much of a rebel, and too good, but you cannot help but love them. 
	\contact{\cYS{}}: Have not met this one yet, but your youngest seems to be infatuated and cannot stop talking about this person. 
	
\end{contacts}

\end{document}
