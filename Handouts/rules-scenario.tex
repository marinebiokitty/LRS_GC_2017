%%%%%
%%
%% This is intended as a nearly complete rules and scenario document
%% that you, the GM, change and complete for your game.  Various
%% comments suggest what parts can be removed or changed based on
%% common game variants.  For example, you will not need the rule for
%% player rooms if your gamespace is not open or otherwise doesn't
%% include living spaces.  Some possibly useful sentences/paragraphs
%% are simply commented out.
%%
%% Feel free to ignore these comments and just write what you want.
%%
%%
%%
%% These rules are merely an example ruleset.  Nothing in them is
%% sacred or even established by consensus.  They do not prescribe a
%% standard.  As a GM, you can change, remove, or rewrite whatever is
%% necessary to fit your game design.
%%
%% However, much of the material presented here, especially in the
%% Getting Started and Items Etc. sections, may be taken for granted
%% by many players.  This has two major implications:
%%
%% 1) Because of gradual shifts, longstanding biases, and first
%% impressions, different players (and GMs!) may have very different
%% assumptions about some details.  Do not brush off the sections that
%% "everyone knows."  Make sure everyone pays attention to the rules
%% as a whole.
%%
%% 2) If you have a clever idea to change some detail in the
%% fundamental parts of the rules, make sure to draw attention to it.
%% For example, if you modify the item bulkiness rules, don't just
%% gloss over your changes in the middle of a paragraph.
%%
%% You may want a section at the end as a reference to any fundamental
%% changes.  See "New Rule Summary" at the end.
%%
%%
%%
%% The martial combat system (and related health states and ranged
%% combat system) in this doc is known as "darkwater," named after The
%% Pirates of Darkwater, for which the first version was written.  It
%% is included as an example combat system.
%%
%%
%%
%% Basic guidelines for rules-writing:
%%
%% Use simple, concise, and precise language.  Avoid colloquialisms
%% and speech mannerisms.  Write in the second person.  Use the
%% imperative voice when possible.  Write like you are writing
%% directions.
%%
%% When you change contexts, like going from writing to the attacker
%% to writing to the defender, at least change paragraphs.
%%
%% When first introducing a major term and/or abbreviation, use bold
%% text.  Try to define terms before you use them.  Combined with
%% section/subsection/paragraph headings, this will make the rules
%% easier to skim for reference.  First establish context, then go
%% into detail.
%%
%% Be thorough, but do not ramble.  Try to make the spirit of the
%% rules clear in addition to the letter.  If the spirit is hard to
%% relay, then the rule may be too complicated.
%%
%%%%%

\documentclass[sheet]{LRSguildcamp1}

%% document-wide tweaks
\interlinepenalty10000
\setstretch{1}
\def\mytype{Rules and Scenario}
\lfoot{}\rfoot{}
\parindent0pt

\begin{document}

%% layout for cover page
\thispagestyle{empty}
\parskip0pt

%% title box
\begin{center}\LARGE\bf\begin{tabular}{|c|}
  \hline \gamename\\ \gamedate\\ Rules and Scenario\\ \hline
\end{tabular}\end{center}

\vfill\vfill

%% player side of the GM/player contract
The following are the rules for {\em\gamename}, a real-time,
real-space roleplaying game sponsored by the \organization{}.
You are responsible for knowing these rules.  Many of them are
nigh-impossible to enforce and rely upon the honor system.  Do not
cheat.  Do not abuse loopholes.  Play fair.  Be your own harshest
critic.

\vfill

%% GM side of the GM/player contract
The {\bf gamemasters} ({\bf GMs}) run the game.  If you have any
problems or questions concerning the game, contact a GM.  Rulings they
make are final.  They may violate the letter of the rules to preserve
the spirit.  The GMs promise to be as fair and reasonable as possible.
Neither they nor these rules are perfect.

\vfill

%% have fun
This game is intended to be fun.  Getting into character, roleplaying,
being dramatic, and playing competitively can all increase the fun of
the game.  Do not take the game too seriously.  Even if you are
losing, keep a good attitude.  When the game is over, the real winners
are the players with the best stories.

\vfill

%% be safe
This is only a game.  Everyone involved should act with courtesy,
sportsmanship, patience, and taste.  The GMs may expel anyone they
believe to be violating the spirit of the rules or the game.  Emotions
may run high.  If you think things are crossing the line from game to
reality too much, or if you are just getting too stressed, calm down
and maybe take a break.  Stay in control.  Use common sense.  Always,
play safely, then play to have fun.

\vfill

%% disclaimer and copyright
%% author list auto-generated from Lists/gm-LIST.tex
This game is a work of fiction.  Although it may refer to things in
the real world, it does so only for the sake of the scenario.  It does
not represent the opinions of the GMs or the \organization{}.
These rules are modifications of those used in previous games.  This
game and all materials thereof are copyright 2017 by Amy Russo, Betty Bong, Louis Wasserman, and the \organization{}.

\vfill\vfill

\begin{center}\bf
  Brought to you by the \organization{}
\end{center}

\vfill

\clearpage

%% layout for Table of Contents page
\thispagestyle{empty}
\tableofcontents

\clearpage

%% layout for main body of rules
\setcounter{page}{1}
\parskip5pt


\section{Scenario}
You live in a world of superhumans, people who have special powers they can use to heroically protect civilians, or to villainously seek power for themselves.
You are all associates of the Kaper family, a legendary family of superhumans led by their matriarch, the renowned supervillain Voil\'a, She has called this family reunion for some kind of special announcement.  This is the first time the entire family has gathered in years, and there is much to discuss…

\emph{Unless you know otherwise, gamespace is open. Your character is free to return to their home city by going to the doorway and declaring your intentions with a 15 count, but you MAY NOT return in character if you choose to leave.}

%% The Scenario should present the setting of the game (including time
%% and place).  It may include basic history and culture.
%% Sufficiently long and/or complicated games might have a full
%% timeline.
%%
%% You may also want to give meta-information like basic roleplaying
%% and costuming hints.

\paragraph{Game Times:} Game runs from 1pm to 4pm on {\bf \gamedate{}}, in the Mitchell Park Library.  PCs are expected to be in-game for the entirety.  Cleanup and Wrap Up will immediately follow the end of
game. Please plan to arrive by {\bf 12:30 pm} to get situated before the game starts. {\bf If you will be late, you must CALL the GMs and let us know. Betty: \gmBetty{\MYphone}} 

\clearpage
\section{Getting Started}

%% Character packets come first, since they are the tangible things
%% handed to players.  Also a convenient place to define Player
%% Character.
\subsection{Character Packets}

Your character packet is a big manila envelope.  It contains your
role: who you are, what you're up to; everything about your part as a
{\bf player-character} ({\bf PC}) in the game.  Read all the contents
and generally keep them with you during the game.  If you are missing
something or find something which doesn't seem to belong to you, tell
one of the GMs.  Character packets are confidential.  Game materials
which cannot be given to other players are marked ``Not
Transferable,'' whereas things which can be given to others are marked
``Freely Transferable'' or ``Game Item.''

Your Character Packet would normally contain:
%% other things your game uses, like money, should be described below

%% no character names on badges, yes to character descriptions
\paragraph{Name-Badge:} A name-badge with your player name, character
description, and {\bf badge number} on it shows that you are in the
game; wear it visibly while you are playing.  It represents your
character's body in-game.  Badge numbers are not in-game information.
See the {\em Character Bodies} and {\em Badge Numbers} sections for
more details.

\paragraph{Character Sheet:} Your character sheet describes who you
are and what you are up to.  It contains a list of everything else
that should be in your character packet.  Do not show or read your
character sheet to other players.

\paragraph{Bluesheets:} A bluesheet describes information common to
members of a group.  When in conflict, character sheet information
overrides bluesheet information.  Do not show or read a bluesheet to
other players.

\paragraph{Greensheets:} A greensheet describes and expands abilities,
mechanics, or in-game knowledge.  Do not show or read a greensheet to
other players.

\paragraph{Stat Card:} Your stat card lists your statistics.  You
might not know what all of your stats mean.  Do not show your stats to
others.  % The reverse side is a {\bf death report}; fill it out and give it to the GMs when your character dies.

\paragraph{Money:} Money is in the form of poker chips.  Each poker chip represents \$10,000 in your bank account.  That money is not actually physically present.  Other players cannot see how much you have or take it from you.

\paragraph{Ability Cards:} An ability card explains a special ability
your character has.  The front side describes the effects; show it to
players when you use the ability.  The reverse is the rules of use and
must not be shown to other players.

\paragraph{Memory/Event Packets:} A memory packet is an envelope or
stapled piece of paper with a {\bf trigger} which describes when to
open and read it.  If the trigger is a number, open the packet when
you see something with that number.  If it's a quoted phrase, open
when you hear or read it in-game.  If it's a symbol, open when
instructed.  Do not take game action based on an unopened trigger.  Do
not show or read a memory packet to other players.

\paragraph{Items:} All in-game items may be transferred from character to
character.  See the {\em Items Etc.} section for more details.


\clearpage
%% Some Assassin Game fundamentals
\subsection{Reality and Game Reality}

There is a big difference between reality and game reality.  Players
must treat each other with courtesy and explain to each other what
their characters perceive in confusing situations; e.g.\ ``My
character's hands are covered in blood,'' an {\bf out-of-game}
statement.  Characters are under no such restrictions, and may do what
it takes to further their goals; e.g.\ ``Uh, hi Bob.  Just got back
from the butcher shop,'' an {\bf in-game} statement.

{\bf Metagaming} is inferring in-game knowledge that is inappropriate
for your character from out-of-game information.  Do your best to not
metagame and especially to prevent the risk of metagaming.  Be your
own harshest critic.

\paragraph{Halts:} A halt pauses game action.  To call one, say ``game
halt'' in a clear and audible voice; other players around a corner
should hear you, but you shouldn't scare some poor grad student.  End
a halt by saying ``three, two, one, resume.''  Call a halt for one of
only three reasons: because a rule instructs you to, for safety and
similar out-of-game issues (see Non-players section below), or to pause game and fetch a GM (which you
should normally avoid doing).

\paragraph{Not-Here:} You may go not-here by turning your name-badge
around so the ``I'm Not Here'' side is showing (or by removing your
badge entirely, if you are leaving game).  Putting a hand on your
head, visible from a distance, helps if you're near other players.  Go
not-here for one of only three reasons: because a rule instructs you
to, to leave game, or to fetch a GM while in a halt (which you should
avoid).

%% last two sentences not for closed-time/space game
When you are not-here, your character is not there.  Your character
cannot see, hear, or remember any game actions or information you (the
player) happen to encounter.  Avoid other characters, common game
areas, game signs, or any sort of game interaction.  
%%To leave or enter game for the night/day/whatever, walk to somewhere public.  Don't go not-here in front of other characters; give them a fair chance to interact with (ambush) you.

%% for shorter, intense games (SIK, etc.), add a NP Halt rule
\paragraph{Non-Players:} Use tact and common sense when dealing with
non-players ({\bf NPs}).  NPs may not knowingly affect the game. 

Avoid conspicuous or threatening game actions in front of NPs.
If, despite your most valiant efforts, some NPs do get upset, call the
GMs who will help calm them down.

If you are about to take an action that would likely upset a nearby NP, you may call a game-halt. This is considered an out-of-game issue.

%% not for closed-space games
%%\paragraph{Player Rooms:} Players may retreat to their rooms to study,
%%sleep, or whatever in safety.  Your character may not enter a player's
%%room unless invited in-game.  This has traditionally been called the
%%``jhereg rule.''  Do not use your room as an impenetrable meeting
%%place or stash site.  If your character is in-game in your room, other
%%characters may interact with (kill, torture) you.  Roommates and
%%similar are considered to have separate rooms for this rule.

\paragraph{Observers:} An observer is someone not playing the game who
has agreed to watch.  They generally wear an observer headband or an
observer name-badge.  Observers have traditionally been called
``ghosts.''  They should stay out of the way; you can always ask an
observer to leave.  If a friend who is not playing wants to observe
game, send them to the GMs.

\clearpage
\paragraph{Mechanics:} Many actions your character can take, such as
walking, talking, and general interaction with other characters, are
represented by you doing them.  Others are performed via
abstract mechanics, which are described in ability cards, greensheets,
and rules.  The abstract information for mechanics (like badge
numbers) may not be discussed in-game.  If you want to do something
special for which there is no mechanic, ask a GM.

Become familiar with your mechanics before game starts, especially
those which occur under time-pressure.  Game action will
not stop for memory packets, greensheets, or such.

A {\bf kludge} (and derivative forms like ``kludge-ite'') is something
impervious to logic and cleverness, usually for game-balance.  You
can't affect a kludge without a specified mechanic.

{\bf Zone of Control} ({\bf ZoC}) is a rough distance measurement.
You are within ZoC of someone if your outstretched fingers can touch
their outstretched fingers.  Double-ZoC is twice this distance,
triple-ZoC is three times, etc.

{\bf Headbands} represent obvious visual effects; wear them visibly on
your head.  If you see a headband and don't know what it represents,
ask.  If you are wearing a headband, tell people what their characters
see. See the end of this document for additional details.

An {\bf interruptible} mechanic has some duration, and may involve
continuous roleplaying.  It is stopped if you are attacked or if
someone within ZoC says {\bf ``I stop you''} or an equivalent phrase.
Some mechanics may be easier or harder to interrupt.

% A {\bf n-count} is an interruptible mechanic with a repeated, counted incant (``I pour a drink one, I pour a drink two, I pour a drink three'').  Speak clearly; each count must take at least a full second. Each n-count will specify the number, e.g.\ a 3-count.

%To play {\bf Rock, Paper, Scissors} ({\bf RPS}), you and your
%opponent(s) say ``one, two, three, show'' in unison.  On ``show''
%everyone displays and compares their chosen symbol.  Rock is a closed
%fist.  Paper is a flat hand with palm down.  Scissors is a fist with
%the first two fingers extended, looking vaguely like a pair of
%scissors.  Rock defeats (crushes) scissors, scissors defeats (cuts)
%paper, paper defeats (covers) rock, and any symbol ties with itself.
%You may see or be able to play other, special symbols; the wielder
%will know what happens.

\paragraph{Safety:} This is a game.  Real violence is unacceptable.
Game action should cause no real-world damage, either to people or
property.  If something dangerous is happening, call a halt.  Stay in
control, use common sense, and do not endanger yourself or others.
You should not run or otherwise force your way into or through someone
else's ZoC, and you should not make physical contact with another
player without permission.

\subsection{Basic Strategy}

Make sure you understand the rules.  If you are completely confused,
get a GM who will try to help you out.  Make sure you know enough
about your character to role-play them when you start talking to
other people.  Read through your entire packet a couple of times, and
skim through it again right before game starts.  If you don't know
something about your character, ask a GM.

As a character, your first priority should be to open lines of
communication.  Contact people, show up at meetings, and chat.  Try to
be easy to get in touch with.  Ask people questions on relevant
subjects.  They'll probably lie, but you may find something out.

There are no guarantees that you can trust anyone, but since
cooperation is the key to accomplishing things, you will be forced to
trust people anyway.  The most trustworthy people are probably those
who need you.

%Those who do not study history are doomed to repeat it. This game has a lot of history in it. You should strive to learn as much as you can about your history, and the history of those around you.


\clearpage
\section{Items Etc.}

Many in-game items are represented by little white cards with a number
and description.  Item cards may be shown to others, passed around,
stolen, etc.  The {\bf item number} on the card is not in-game
information and may not be discussed.  

Some mechanics in game may involve making item cards. Such items should be clearly marked as ``in game'' and treated as such.

Use common sense.  You can't carry a hundred rocks in your pocket,
fold a sword in half, or hide a life-sized statue in a fire hose.  You
can't stop a bullet with a set of blueprints or rip apart a metal safe
with your bare hands.  Even if your bag can carry a shovel in it, the
shovel noticeably sticks out (``you see a shovel sticking out of my
bag'').

\paragraph{Written Information:} If you write in-game information down
on a piece of paper, that paper is now an in-game item and must be
clearly marked as such.  Don't write in-game information on
out-of-game documents (character sheet, etc.).  Don't write
out-of-game information (like memory packet triggers) on in-game
documents.

\paragraph{Envelopes:} Some items and locations may have an attached
envelope (or just be a labeled packet or folded paper).  The envelope
may include directions for when to open these (``open packet if you
press the big red button'' or ``open packet if you eat this'');
otherwise you may only open them if instructed.  Close them when you
are done.  Open and close packets gently.

\paragraph{Signs:} Some locations and other game materials are
represented by signs or packets posted throughout game area.  You may
read any signs and must follow any rules printed on them.  If a sign
or packet doesn't have some sort of in-game description (it only has
out-of-game mechanics information, like a number or just a colored
dot), then your character doesn't even see it or know that anything
unusual is there.

\paragraph{Bulkiness:} A bulky item is too big or heavy to be carried
or concealed freely.  Bulkiness is measured in {\bf hands} or {\bf
dots} (how many hands it takes to carry it).  If you are carrying a
bulky item, make it clear to onlookers (hold the card).  A hand
carrying a bulky object may do nothing else.  With one hand less than
required, you may drag a bulky item at a slow pace.  For example, a phone or a laptop might be one hand bulky, a large painting might be two hands bulky, and a couch might be three hands bulky.  You can carry a phone or painting at normal speed, but dragging a couch will be slow.

%\paragraph{Valuable:} Some items are marked ``valuable''. Some plots may require you to acquire valuable items. Any item that has this tag qualifies.

\paragraph{Props:} Some items may have props (physical representations
or {\bf physreps}) associated with them.  The card and physrep should
be kept together.  If they are separated, the card is the real item.
Prop items are as bulky as the physrep.  They can be carried in bags
that can hold them, on straps that are attached to them, etc.

\paragraph{Unstashable Items:} Unstashable items can't be hidden or left behind.  They look too important, valuable, or interesting; NPCs will not let them stay there.  These include any item that has a physrep. This is a kludge.  If you're not leaving an unstashable item in another PC's care, and you want to leave it behind, give it to a GM or observer.  You may leave it in plain sight in a public area if there are other PCs around.

\clearpage
\subsection{Searching, Stashing, and Stealing}

\paragraph{Places:} To search a place, search it.  Normal items can be
stashed in any reasonable, legal place.  Don't put items behind locked
doors, inside ceilings, in construction sites, or in hacking
locations; consequently, don't go rummaging through such places for
game items.  Don't stash or search in places that are not in-game; see
the {\em Game Areas} section for more information.


% TODO: do we want to explain stealing?
Some characters in this game can steal from other characters.  This mechanic involves surreptitiously putting stickers on unaware players' backs.  If you see a player attempting to put a sticker on another player, your character sees this as e.g. reaching into their pockets.  

\clearpage
\section{Violence, Damage, and Death}


This game has no PvP combat, though wounds can ensue if you go into dangerous areas of game.  When wounded, wear a red headband and walk very slowly.  Wounding only lasts for a short period of time in this game.  If a player is wounded, this is visible to anyone who sees them.

\clearpage
\section{Miscellaneous}

%\paragraph{Badge Numbers:} The first digit of your badge number is your character's apparent age in decades. All hands are 2 hands bulky.
%  The second digit is your
%character's apparent burliness: a ``3'' is pretty skinny, a ``5'' is
%average, and an ``8'' is huge and muscular.

%% classic open-time 10-day game times
%\paragraph{Game Times:} Game runs from 8pm on Friday to noon on
%Sunday.  Surviving PCs are expected to be in-game for the entirety.
%Game may end early.  Cleanup and Wrapup will immediately follow the
%end of game.

%% basic open-space game area rules
%\paragraph{Game Areas:} Most publicly-accessible areas on campus are
%considered in-game (your character can move about freely in them).  As
%usual, avoid places it is illegal for you to go, areas under
%construction, etc.  Don't take game actions in bathrooms, private
%offices, activity offices, and other places not all players would be
%allowed to enter.

%When in living areas, such as dorms, remember the {\em Player Rooms}
%section.  Many living areas on campus are not technically accessible
%to all players.  Whether or not to take game action in your living
%area is left to player judgment.

%% if you have restricted-access buildings
%There are some areas on campus that are not publicly in-game.  You may
%not enter them in-character unless explicitly instructed to; if you
%happen to be in them your character is not there.  These areas are:

%% really basic locations?
%The {\bf GM Control Room} is room x-xxx.  You may leave personal items
%with the GMs.  The {\bf Common Room} is room y-yyy.  Do not leave
%food, trash, or personal items in the Common Room overnight.

%% Electronic communication goes hand-in-hand with cluster rules.
%% Specifics of various types of communcation var by genre and game.

%% if game has no athena/phones
%Game action is not allowed in Athena clusters.  Don't hide in them,
%either.  You may not use Athena or phones for any in-game purpose.

%% or even
%\paragraph{Electronic Information:} You may use email, zephyr, IM,
%phones, and other forms of electronic communcation freely for game
%purposes.  You may not violate any rules of use of these devices (no
%packet sniffing, wiretapping, etc.).  When searching a character or
%their stuff, you do not get access to their electronics, except in
%specified instances.  Game action is allowed in Athena clusters, as
%long as you obey the NP rules and don't make a mess.


%% use this section for restating any new rules (big or small) that
%% need attention drawn to them.  It is unnecessary if all new things
%% have their own sections, etc.
%\section{New Rule Summary}
%
%\begin{itemz}
%
%\item Bodies are three hands bulky, not two.
%
%\end{itemz}

\clearpage

%\section{The Neptune Ball Specific Changes}
%This section is a recap of the changes specific to this game from other MIT Assassins Guild style games played at Stanford in the recent past. There are also several new and important game-wide mechanics. Familiarize yourself with them before game.

\subsection{Tape on the floor}
There may be tape on the floor in this game. It represents walls.  You cannot cross tape directly unless you can walk through walls.  You cannot see through tape.  You cannot hear through tape.  (There may be other representations of walls, in addition to the tape; they all behave the same.)

\subsection{The Dinner}
There will be a sit-down dinner that starts 2.5 hours into game. All characters are expected to attend. 

%\subsection{Magical Effects}
%Magical effects in game are represented as item cards labeled ``magical effects.'' These items cannot be revealed with a normal search and are considered {\bf non-transferable} unless you know otherwise.

%\subsection{NPCs}
%The Neptune Ball has many NPC pages running around. They will be wearing blue headbands. They can carry a simple message for you to another player to the effect of: ``So-and-so wants to talk with you. I saw them in X location last.'' They may not be terribly reliable or timely however. Pages cannot carry items unless you know otherwise. NPCs will also spread game-wide announcements, and may play certain additional NPCs as necessary for some mechanics.  {\bf Pages cannot be sent out of a room for any reason. Pages cannot be attacked or killed.}


\subsection{Phones}
Each character is carrying a cellphone.  You may place a call at any time by holding your hand up to your ear like a phone and calling a GM over.  If all GMs are occupied, we appreciate your patience as your call may be put on hold until a GM is free.

\subsection{PR}
A major mechanic in this game is public opinion of heroes and villains.  During game there will be a prominently displayed sign describing the current public opinion of heroes versus villains, measured as a score from 0 to 10.  5 represents neutrality.  Higher scores represent higher public opinion of heroes, and lower scores represent higher public opinion of villains.  There will also be publicly displayed news stories, blog posts, and press releases about information released during game.

\subsection{Stickers}
Placing stickers on another player represents a sketchy action like pickpocketing. If you see someone placing a sticker, you should probably ask what you see. Stickers already in place are out-of-game information.

If there is a sticker on an item or sign, with a time written on it, that item or sign is unusable until that time.  A purple sticker indicates the item is frozen in ice.  A yellow sticker indicates the item has shorted out.

\subsection{Headbands}

Headbands represent the following conditions:
\begin{itemize}
\item Blue: the character is invisible.  You cannot see them, though you can recognize their voice if they speak.
\item Red: the character is (visibly) wounded.
\item Purple: the person is an NPC.
\end{itemize}

\subsection{Costumes}

This is a game with superheroes and supervillains.  Costuming is optional, as always, but players may choose to wear costumes with capes, masks, and other accoutrements.  Costumes do not have any in-game effects.

\subsection{Elaborating on Character Background}
At points during game, you may be asked to elaborate on part of your character's backstory that has not been specified in your character sheet.  Please feel free to make up these details without checking in with the GMs.


\subsection{Closing Notes}

These rules are imperfect.  The GMs may violate the letter of the
rules to preserve the spirit.  We hope these rules are reasonably
clear, but if you have any doubts about your interpretation, talk it
over with us in advance. 

We should also add, as much as we hate to
admit it, we GMs are human: when all of our carefully laid plans are
going haywire, we may lose our cool.  The best way to deal with people
is remaining calm and friendly, especially when everyone is tired and
hungry.

We hope you have lots of fun.  Good luck.

\end{document}



